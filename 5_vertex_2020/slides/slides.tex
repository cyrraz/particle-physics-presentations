%===========================================================================================================
\begin{frame}{Outline}
 \tableofcontents%[currentsection, currentsubsection]
\end{frame}
%===========================================================================================================
\section{Introduction}
%===========================================================================================================
\newcommand*{\talk}[4]{{#1}, \emph{#2}, \href{#3}{\color{blue!40!gray} [#4]}.}
\newcommand\itemshape[1]{\setbeamertemplate{itemize item}[#1]\usebeamertemplate{itemize item}}
\begin{frame}{Belle II talks and posters at VERTEX 2020}
\bi
{\small
\item \talk{Qingyuan Liu}{Operational Experience and Performance of the Belle II Pixel Detector}{https://indico.cern.ch/event/895924/contributions/3968848/}{A02}
\itemiii {\talk{Giuliana Rizzo}{The Belle II Silicon Vertex Detector: Performance and Operational Experience in the first year of data taking}{https://indico.cern.ch/event/895924/contributions/3968849/}{A03}}
\itemiii {\talk{Yuma Uematsu}{A Study for Hit-time Reconstruction of Belle II Silicon Vertex Detector}{https://indico.cern.ch/event/895924/contributions/3988295/}{P02}}
\itemiii {\talk{Sagar Hazra}{Particle identification in Belle II silicon vertex detector}{https://indico.cern.ch/event/895924/contributions/3993222/}{P03}}
\itemiii {\talk{Katsuro Nakamura}{Development of the thin and fine-pitch silicon strip detector aiming for the Belle II upgrade}{https://indico.cern.ch/event/895924/contributions/3988300/}{P05}}
{\setbeamercolor{item}{fg=red}\itemiii {\talk{Cyrille Praz}{Tracking performance and interaction point properties at the Belle II experiment}{https://indico.cern.ch/event/895924/contributions/4018211/}{A07}}}
\itemiii {\talk{Jerome Baudot}{Upgrade of the vertex detector of the Belle II experiment}{https://indico.cern.ch/event/895924/contributions/3968854/}{B01}}
}
\ei
\end{frame}
%===========================================================================================================
\begin{frame}{SuperKEKB}
\begin{columns}
\begin{column}{0.48\linewidth} 
\bi \small
\item{\epem collider.}
\itemi{$\sqrt{s}=10.6\gev=\mathrm{m}(\Y4S)c^2$.}
\itemi{$\mathrm{BR}(\Y4S\to\BB)>96\%$.}
\itemi{$\displaystyle\int_{2019}^{\rm present}\,L\,\mathrm{dt}\approx70\invfb$.}
\itemi{World highest instant. luminosity.}
\bi
\item{$L=2.4\times10^{34}\cms$ achieved in June 2020.}
\ei
\ei
\end{column}
\begin{column}{0.52\linewidth}
\centering
\includegraphics[width=0.75\linewidth]{figs/super_kekb}
\includegraphics[width=0.65\linewidth, angle=-90]{figs/SuperKEKB_roadmap2020}
\end{column}
\end{columns}
\end{frame}
%===========================================================================================================
\begin{frame}{Nano-beam scheme (idea from Pantaleo Raimondi)}
\begin{columns}
\begin{column}{0.52\linewidth} 
{\small
\bi
\item Goal: $\beta_y^*=0.3\mm$.
\item Hourglass effect limited if $\sigma^{eff}_z<\beta_y^*$.
\item Half crossing angle:
\bi
\item $\phi_x\approx40\mrad$.
\ei
\item Nominal beam spot parameters:
\bi
\item $\sigma_x\approx10\mum$.
\item $\sigma^{eff}_z=\frac{\sigma_x}{\sin\phi_x}\approx0.25\mm$.
\item $\sigma_y\approx50\nm$.
\ei
\ei
\includegraphics[width=0.9\linewidth]{figs/nano_beam_2} 
}
\end{column}
\begin{column}{0.48\linewidth}
\includegraphics[width=1\linewidth]{figs/nano_beam} 
\small\href{https://docs.belle2.org/record/1212?ln=en}{\color{blue!40!gray} [BELLE2-TALK-CONF-2018-142]}
\href{https://arxiv.org/abs/1809.01958}{\color{blue!40!gray} [1809.01958]}
\end{column}
\end{columns}
\end{frame}
%===========================================================================================================
\begin{frame}{Crab waist transformation}
\bi
\item Since 2020, SuperKEKB has introduced the crab waist transformation to further increase luminosity. {\small \href{https://arxiv.org/abs/physics/0702033}{\color{blue!40!gray} [0702033]}, \href{https://arxiv.org/abs/1608.06150}{\color{blue!40!gray} [1608.06150]}}
\bi
\itemi Align $\min(\beta_y)$ of one beam along the trajectory of the other beam.
\itemi Additional benefits are lower background and improved beam lifetime.
\ei
\ei
\centering
\begin{tikzpicture}
\node (a) {\includegraphics[width=0.4\linewidth]{figs/crab_waist_1}};
\node (b) [right=1.5cm of a] {\includegraphics[width=0.4\linewidth]{figs/crab_waist_2}};
\draw[thick, ->] (a.east) to (b.west);
\end{tikzpicture}
\end{frame}
%===========================================================================================================
\begin{frame}{Importance of the impact parameter resolution}
\bi
\item SuperKEKB has a smaller Lorentz boost compared to KEKB.
\itemiii Belle II has a better impact parameter resolution (factor $\sim1.5-2$) thanks to its pixel detector and a smaller beam pipe diameter.
\bi
\itemiii {$(\beta\gamma)_{\rm SuperKEKB}\approx0.28<0.42\approx(\beta\gamma)_{\rm KEKB}$.}
\itemiii Precise measurements of decay vertices is crucial for time-dependent studies.
\ei
\ei
\centering
\includegraphics[width=0.7\linewidth]{figs/bsig_btag}
\bi
\item $\B$ mesons nearly at rest in the \Y4S frame. 
\bi
\itemii{$\Delta t \approx \frac {\Delta z_{\rm {boost}}} {c (\beta \gamma)_{\Y4S}},\hspace{0.5cm} \Delta z_{\rm {boost}}=\mathcal{O}(100\mum)$.}
\ei
\ei
\end{frame}
%===========================================================================================================
\begin{frame}{Proper-time resolution}
\bi
\item Thanks to the Belle II vertex detector, the proper-time resolution is a factor $\sim2$ better than at Belle and Babar.
\itemii Example of a $D$ meson lifetime measurement {\tiny \href{https://docs.belle2.org/record/2074?ln=en}{\color{blue!40!gray} [BELLE2-TALK-CONF-2020-046]}}:
\ei
\vspace{0.5cm}
\centering
\includegraphics[width=0.5\linewidth]{figs/time_resolution} 
\end{frame}
%===========================================================================================================
\section{Belle II tracking}
%===========================================================================================================
\begin{frame}{\only<1>{The Belle II detector} \only<2>{Tracking subdetectors}}
\begin{columns}
\begin{column}{0.45\linewidth} 
\begin{itemize} \small
\item  \color<2>{red} Pixel Detector (PXD).
\vspace{0.25cm}
\item Silicon Vertex Detector (SVD).
\vspace{0.25cm}
\item Central Drift Chamber (CDC).
\vspace{0.25cm}
\item \color<2>{gray}Calori\color<2>{red}meter \color<2>{gray} (ECL).
\vspace{0.25cm}
\item Aerogel Ring-Imaging Cherenkov (ARICH).
\vspace{0.25cm}
\item Time-Of-Propagation (TOP) counter.
\vspace{0.25cm}
\item $K_L^0$ and $\mu$ detection (KLM).
\end{itemize}
\end{column}
\begin{column}{0.55\linewidth}
\centering
\begin{tikzpicture}
\node (a) {\includegraphics[width=0.9\linewidth]{figs/belle2_with_flags}};
\node (b) [below=0.25cm of a] {\includegraphics[width=0.65\linewidth]{figs/vxd.pdf} };
\node(c)  [above=-0.25cm of b, xshift=-1.5cm]{};
\draw[thick, red,->] (a.south west){}+(0.9cm,0.1cm) to (c);
\end{tikzpicture}
\end{column}
\end{columns}
\end{frame}
%===========================================================================================================
\begin{frame}{VXD (=PXD+SVD)}
\begin{columns}
\begin{column}{0.5\linewidth}
\hspace{4.cm}\includegraphics[width=0.3\linewidth]{figs/pxd1}
\begin{itemize} \small
\vspace{-1.8cm}
\item PXD.
\vspace{0.05cm}
\begin{itemize}
\item Pixel detector. 
\vspace{0.05cm}
\item 2 layers.
\vspace{0.05cm}
\item Radii: 14, 22\mm.
\vspace{0.1cm}
\end{itemize}
\item SVD.
\vspace{0.05cm}
\begin{itemize}
\item Double-sided silicon strips.
\vspace{0.05cm}
\item 4 layers.
\vspace{0.05cm}
\item Radii: 39 to 135\mm.
\end{itemize}
\end{itemize}
\end{column}
\begin{column}{0.5\linewidth}
\includegraphics[width=1.0\linewidth]{figs/vxd.pdf}
\end{column}
\end{columns}
\includegraphics[width=1.0\linewidth]{figs/FullVXD_transverse}
\end{frame}
%===========================================================================================================
\begin{frame}{CDC}
\begin{columns}
\begin{column}{0.45\linewidth} 
\begin{itemize} \small
\item Drift chamber.
\vspace{0.25cm}
\item $\approx50\,000$ wires.
\vspace{0.25cm}
\item 56 layers.
\vspace{0.15cm}
\begin{itemize}
\item Radii: 168 to 1111.4\mm.
\vspace{0.25cm}
\end{itemize}
\item 9 superlayers.
\vspace{0.15cm}
\begin{itemize}
\item axial orientation (A).
\vspace{0.15cm}
\item stereo orientation (U,V).
\vspace{0.25cm}
\end{itemize}
\item Configuration:
\vspace{0.15cm}
\begin{itemize}
\item AUAVAUAVA.
\end{itemize}
\end{itemize}
\end{column}
\begin{column}{0.55\linewidth}
\centering
\includegraphics[width=0.7\linewidth]{figs/cdc_wires}
\includegraphics[width=0.7\linewidth]{figs/RPhiSlice}
\end{column}
\end{columns}
\end{frame}
%===========================================================================================================
\begin{frame}{Charged particles from simulated \Y4S events}
\begin{columns}
\begin{column}{0.5\linewidth} 
\begin{itemize} 
\item $p_{\mathrm{T}}<100\mevc$:
\vspace{0.2cm}
\begin{itemize}
\item Track does \emph{not} reach CDC.
\item Detected by standalone SVD.
\vspace{0.4cm}
\end{itemize}
\item $p_{\mathrm{T}}\in[100,\,300]\mevc$:
\vspace{0.2cm}
\begin{itemize}
\item Track can curl inside CDC.
\vspace{0.5cm}
\end{itemize}
\end{itemize}
\adjustbox{max height=0.9\textheight,max width=0.9\textwidth}{
 \centering
\input{figs/tikz_particle_pie_chart.tex}
}
\end{column}
\begin{column}{0.5\linewidth}
\includegraphics[width=1\linewidth]{figs/transMomentumDistribution}
\vspace{0.5cm}
\hfill
\href{https://www.sciencedirect.com/science/article/pii/S0010465520302861}{\tiny \color{blue!40!gray}[doi: 10.1016/j.cpc.2020.107610]}
\end{column}
\end{columns}
\end{frame}
%===========================================================================================================
\begin{frame}{Track reconstruction steps\\ {\tiny Belle II Tracking Group, \emph{Track finding at Belle II}, Computer Physics Communications (2020). \href{https://www.sciencedirect.com/science/article/pii/S0010465520302861}{\color{blue!40!gray}[doi: 10.1016/j.cpc.2020.107610]}}}
\centering
\adjustbox{max height=0.8\textheight,max width=1.0\textwidth}{
% Author: Frantisek Burian
\documentclass{standalone}
\usepackage{tikz}
\usetikzlibrary{arrows, arrows.meta, decorations.markings, shapes, positioning, backgrounds, calc, fit, shapes.geometric}

\tikzstyle{pathelement} = [rectangle, draw, text width=10em, minimum height=1.5em, text centered, line width=0.3mm]

\tikzstyle{vecArrow}[black] = [
    #1,
    thick, 
    double distance=1.4pt, 
    shorten <= -0.85pt,
    arrows={-Triangle[angle=60:7pt,#1,fill=white]},
    postaction = {draw, -, line width = 1.4pt, white, shorten >= 4.0pt, shorten <= 4.0pt}
  ]
\definecolor{chapter-color}{rgb}{.27,.39,.67}
\begin{document}

\begin{tikzpicture}
    \node[pathelement, fill=chapter-color!80!white, text width=12em] at (1.5, 1.5) (background) {CDC Background Filter};
    \node[pathelement, fill=chapter-color!80!white, text width=6em] at (0, 0) (legendre) {CDC Legendre};
        \node[pathelement, fill=chapter-color!80!white, text width=6em] at (3, 0) (ca) {CDC Cellular Automaton\vphantom{g}};
        
    \node[pathelement, fill=chapter-color!80!white, text width=6em] at (1.5, -1.5) (merger) {Merging};
        \node[pathelement, fill=chapter-color!50!white, text width=6em] at (7.5, -1.5) (svd_ckf) {SVD CKF\vphantom{g}};

        \node[pathelement, fill=chapter-color!50!white, text width=6em] at (7.5, -3.26) (svd_vxdtf2) {SVD Standalone\vphantom{g}};
        \node[pathelement, fill=chapter-color!50!white, text width=6em] at (1.5, -3.25) (ckf_merger) {Combined Fit};
        
        \node[pathelement, fill=chapter-color!30!white, text width=6em] at (3, -4.75) (pxd) {PXD CKF};
        
    \node[pathelement, fill=chapter-color!10!white, text width=6em] at (4.5, -6.25) (fit) {Track Fit};
        
        \draw[vecArrow] (background) -- (legendre);
        \draw[vecArrow] (background) -- (ca);
        \draw[vecArrow] (legendre) -- (merger);
        \draw[vecArrow] (ca) -- (merger);
        \draw[vecArrow] (merger) -- node[pathelement, above=0.25, midway, text width=6em] {CDC Tracks} (svd_ckf);
        \draw[vecArrow] (svd_ckf) -- node[pathelement, right=0.25, midway, text width=10em] {Remaining SVD Hits} (svd_vxdtf2);
        \draw[vecArrow] (svd_vxdtf2) -- node[pathelement, above=0.25, midway, text width=6em] {SVD Tracks} (ckf_merger);
        \draw[vecArrow] (merger) -- node[pathelement, left=0.25, midway, text width=6em] {CDC Tracks} (ckf_merger);
        \draw[vecArrow] (ckf_merger) -- (pxd);
        \draw[vecArrow] (pxd) -- (fit);
\end{tikzpicture}

\end{document}

}
\end{frame}
%===========================================================================================================
\begin{frame}{SVD Standalone \hfill {\small \href{https://www.sciencedirect.com/science/article/pii/S0010465520302861}{\color{blue!40!gray}[doi: 10.1016/j.cpc.2020.107610]}}}
\bi
\item Each SVD sensor is divided in $3\times3$ sectors. 
\itemiii Sector map trained on simulated tracks to learn geometrical relations between sectors.
\itemiii Filters reject bad space point combinations (angle, timing).
\itemiii Cellular automaton yields a set of paths.
\itemiii Track candidates selected based on $\chi^2$ and \#degrees of freedom.
\ei
\centering
\includegraphics[width=0.3\linewidth]{figs/sectorMapSectors}  
\end{frame}
%===========================================================================================================
\begin{frame}{Standalone SVD to CDC extrapolation}
\bi
\item Extrapolating tracks found by the standalone SVD to the CDC improves the CDC track-finding effiency.
\bi
\itemii Significant improvement at large $|\tan\lambda|$.
\ei
\itemii Addition of CDC hits improves momentum resolution of the full track. 
\ei
\vspace{0.25cm}
\centering
2-track events (data), $\lambda\equiv\frac{\pi}{2}-\theta$\\
\includegraphics[width=0.6\linewidth]{figs/efficiency} 
\end{frame}
%===========================================================================================================
\section{Beam spot parameter measurement}
%===========================================================================================================
\begin{frame}{Outline}
\tableofcontents[currentsection, currentsubsection]
\end{frame}
%===========================================================================================================
\begin{frame}{Track parametrisation: 2D picture}
\begin{itemize}
\item {\color{cyan}Blue area}: high and low energy beams overlap.
\vspace{0.2cm}
\item $\phi_0$: azimutal angle at the point of closest approach (POCA).
\vspace{0.2cm}
\item $d_0$: Transverse impact parameter at the POCA.
\vspace{0.2cm}
%\vspace{1.0cm}
\end{itemize}
\centering
\includegraphics[width=0.6\linewidth]{figs/d0_definition.pdf}
\begin{itemize}
\itemi {$\sigma_{68}(d_0)$ depends on the intrinsic detector resolution $\sigma_i$ and the beam spot size:}
\vspace{0.25cm}
\begin{itemize}
\item $\sigma_{68}(d_0)=\sqrt{\sigma_i^2+(\sigma_x\sin\phi_0)^2+(\sigma_y\cos\phi_0)^2}$.
\vspace{1.0cm}
\end{itemize}
\end{itemize}
\end{frame}
%===========================================================================================================
\begin{frame}{Beam spot size and impact parameter resolution  (2019) }%\hspace{1cm} \href{https://docs.belle2.org/record/1511}{\tiny [BELLE2-NOTE-TE-2019-01]}}
\begin{columns}
\begin{column}{0.5\linewidth} 
\begin{itemize} 
\vspace{-0.4cm}
\item \small Select tracks in 2-track events.
\vspace{0.2cm}
\end{itemize}
\adjustbox{max height=0.9\textheight,max width=0.9\textwidth}{
 \centering
\begin{tabular}{lllll}
    \hline
    Variable					& & Requirement 	& & Unit  \Tstrut\Bstrut\\ 
    \hline
    $|d_0|$						& & $<3$ 		& & \mm\Tstrut\\
    $|z_0|$						& & $<1$ 		& & \cm \\ 
    \# selected tracks in the event	& & $=2$ 		& & \\
    $p_\mathrm{T}$				& & $>1$ 		& & \gevc \\ 
    $|\theta_0-\pi/2|$				& & $<0.5$  		\\  
    $p\beta\sin(\theta_0)^{3/2}$	& & $>2$		& & \gevc \\  
    \# PDX hits					& & $\ge1$  		\\
    \# SVD hits					& & $\ge8$ 		\\
    \# CDC hits					& & $>20$		\\   
    \# selected tracks in the event & & $=2$      \\
    product of the charges in the event & & $<0$ & & $C^2$ \Bstrut\\
    \hline
  \end{tabular}
  }
\hfill
\vspace{0.2cm}
\includegraphics[width=0.9\linewidth]{figs/d0_definition} 
\begin{itemize} 
\item \small $\sigma_{68}$(.): half of the symmetric range around the median containing 68\% of the distribution.
\end{itemize}
\end{column}
\begin{column}{0.45\linewidth}
\includegraphics[width=1\linewidth]{figs/d0_width} 
\includegraphics[width=1\linewidth]{figs/d0_difference_width} 
\end{column}
\end{columns}
\end{frame}
%===========================================================================================================
\begin{frame}{Beam spot size measurement: 1st method  (2019) }%\hspace{1cm} \href{https://docs.belle2.org/record/1511}{\tiny [BELLE2-NOTE-TE-2019-01]}}
\bi
\item Unfolding of the beam profile.
\ei
\begin{tikzpicture}
\node (a) {\includegraphics[width=0.4\linewidth]{figs/d0_width}};
\node (a2) [below=0.1cm of a.north east]{$2$};
\node (b) [right=0.25cm of a]{$-$};
\node (c) [right=0.25cm of b] {\includegraphics[width=0.4\linewidth]{figs/d0_difference_width}};
\node (c2) [below=0.1cm of c.north east]{$2$};
\node (d) [right=0.25cm of c]{$=$};
\coordinate (a0) at ($(a.south west) +(-0.25cm,2.5cm)$);
\draw[thick] (a0) -- (a.south west) -- (a.north west)+(0.0cm,0.0cm) -- (c.north east);
\end{tikzpicture}
\centering
\includegraphics[width=0.4\linewidth]{figs/beam_profile} 
\end{frame}
%===========================================================================================================
\begin{frame}{Beam spot size measurement: 2nd method (2020)}
\bi
\itemi The beam spot size can be obtained from the correlation of the impact parameters $d_1$ and $d_2$ in 2-track events.\footnote{\tiny S. Donati and L. Ristori, \emph{Measuring beam width and SVX impact parameter resolution}, Tech. rep. CDF-Note-4189, Batavia, USA: FERMILAB, 2003, \href {http://beamdocs.fnal.gov/AD-public/DocDB/ShowDocument?docid=609}{\color{blue!40!gray} [url]}. See also \href{https://arxiv.org/pdf/1405.6569.pdf}{ \color{blue!40!gray} [1405.6569]}.}
\bi
\itemiii Beam spot size and resolution separated $\implies$ no unfolding needed.
\ei
\itemii {$d_1=x_{\rm IP}\sin\phi_1-y_{\rm IP}\cos\phi_1$}
\itemii {$d_2=x_{\rm IP}\sin\phi_2-y_{\rm IP}\cos\phi_2$}
\itemii {$d_1d_2=\left(x_{\rm IP}\sin\phi_1-y_{\rm IP}\cos\phi_1\right)\cdot\left(x_{\rm IP}\sin\phi_2-y_{\rm IP}\cos\phi_2\right)$}
\itemii {\small $\left\langle d_1d_2 \right\rangle=\sigma_x^2\sin\phi_1\sin\phi_2+\sigma_y^2\cos\phi_1\cos\phi_2+C\cdot(\sin\phi_1\cos\phi_2+\sin\phi_2\cos\phi_1)$.}
\bi
\item {\vspace{-0.1cm}$C=-\left\langle x_{\rm IP}y_{\rm IP} \right\rangle$.}
\ei
\ei
\end{frame}
%===========================================================================================================
\begin{frame}{Beam spot size measurement: 2nd method (2020)}
\bi
\item {\small $\left\langle d_1d_2 \right\rangle-\underbrace{\sigma_y^2\cos\phi_1\cos\phi_2+C(\sin\phi_1\cos\phi_2+\sin\phi_2\cos\phi_1)}_{D}=\sigma_x^2\sin\phi_1\sin\phi_2$.}
\vspace{0.5cm}
\ei
\begin{columns}
\begin{column}{0.3\linewidth}
\centering
Linear fit\\
\vspace{0.5cm}
\includegraphics[width=0.9\linewidth]{figs/fit_xsize} 
\end{column}
\begin{column}{0.7\linewidth}
\centering
Time dependence of $\sigma_x$\\
\vspace{0.5cm} 
\includegraphics[width=1\linewidth]{figs/xsize} 
\end{column}
\end{columns}
\end{frame}
%===========================================================================================================
\begin{frame}{Beam spot orientation}
\bi
\item {Also possible to measure the beam spot orientation w.r.t. the Belle II coordinate system.}
\vspace{0.5cm}
\ei
\begin{columns}
\begin{column}{0.2\linewidth}
\centering
\includegraphics[width=0.9\linewidth]{figs/tilt_illustration} 
\end{column}
\begin{column}{0.8\linewidth}
\centering
Time dependence of $\theta_{xz}$\\
\vspace{0.5cm}
\includegraphics[width=1\linewidth]{figs/tilt} 
\end{column}
\end{columns}
\end{frame}
%===========================================================================================================
\begin{frame}{Beam spot parameters}
\bi
\item The tracking group is currently developing an algorithm to fully measure the beam spot average position and 3D profile.
\itemi Comparison with accelerator values allows to cross-check consistency. 
\vspace{0.5cm}
\ei
\centering
\input{figs/tikz_feedback.tex}
\end{frame}
%===========================================================================================================
\section{Conclusion and outlook}
%===========================================================================================================
\begin{frame}{Conclusion and outlook}
\bi
\item SuperKEKB has a smaller boost compared to KEKB, but Belle II compensates with a better impact parameter resolution thanks to its pixel detector.
\itemi Beam spot parameters can be measured from the Belle II tracking.
\bi
\itemii Comparison with accelerator values allows for cross-checks.
\itemii The results can be used to improve simulation.
\ei
\ei
\end{frame}
%===========================================================================================================
\section*{Questions}
%===========================================================================================================
\begin{frame}{Questions}
\centering
\Large Thank you for your attention.
\end{frame}
%===========================================================================================================
\section*{Backup} % [noframenumbering]
%===========================================================================================================
\begin{frame}[noframenumbering]{Track parametrisation: full picture \hspace{1cm}
\href{https://arxiv.org/abs/1901.11198}{\color{blue!40!gray}{\small [1901.11198],}
\href{https://github.com/ndawe/tikz-track} {\small [github]}}
}
\centering
\includegraphics[width=0.9\linewidth]{figs/helix3d}
\end{frame}
