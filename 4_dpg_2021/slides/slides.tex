%===========================================================================================================
\def\bdto {BDT$_1$\xspace}
\def\bdtt {BDT$_2$\xspace}
\def\bdtoff {BDT$_{\mathrm{c}}$\xspace}

\def\lumiyfours         {62.9\,\invfb\xspace} % 62.93
\def\lumioff            {9.2\,\invfb\xspace} % 9.22
%===========================================================================================================
\begin{frame}{Outline}
 \tableofcontents
\end{frame}
%===========================================================================================================
\section{Introduction}
%===========================================================================================================
\subsection{The SuperKEKB collider}
%===========================================================================================================
\begin{frame}{SuperKEKB}
\begin{columns}
\begin{column}{0.48\linewidth} 
\bi \small
\item{\epem collider in Tsukuba, Japan.}
\itemi{$\sqrt{s}=10.6\gev=\mathrm{m}(\Y4S)c^2$.}
\itemi{$\mathrm{BR}(\Y4S\to\BB)>96\%$.}
\itemi{$\displaystyle\int_{2019}^{\rm Summer\,\,2020}\,L\,\mathrm{dt}\approx63\invfb$.}
\itemi{World highest instant. luminosity.}
\bi
\item{$L=2.4\times10^{34}\cms$ achieved in June 2020.}
\ei
\ei
\end{column}
\begin{column}{0.52\linewidth}
\centering
\includegraphics[width=0.75\linewidth]{figs/super_kekb}
\includegraphics[width=0.65\linewidth, angle=-90]{figs/SuperKEKB_roadmap2020}
\end{column}
\end{columns}
\end{frame}
%===========================================================================================================
\subsection{The Belle II detector}
%===========================================================================================================
\begin{frame}{{The Belle II detector}}
\begin{columns}
\begin{column}{0.45\linewidth} 
\begin{itemize} \small
\item Pixel Detector (PXD).
\vspace{0.25cm}
\item Silicon Vertex Detector (SVD).
\vspace{0.25cm}
\item Central Drift Chamber (CDC).
\vspace{0.25cm}
\item Calorimeter (ECL).
\vspace{0.25cm}
\item Aerogel Ring-Imaging Cherenkov (ARICH).
\vspace{0.25cm}
\item Time-Of-Propagation (TOP) counter.
\vspace{0.25cm}
\item $K_L^0$ and $\mu$ detection (KLM).
\end{itemize}
\end{column}
\begin{column}{0.55\linewidth}
\centering
\begin{tikzpicture}
\node (a) {\includegraphics[width=0.9\linewidth]{figs/belle2_with_flags}};
\node (b) [below=0.25cm of a] {\includegraphics[width=0.65\linewidth]{figs/vxd.pdf} };
\node(c)  [above=-0.25cm of b, xshift=-1.5cm]{};
\draw[thick, red,->] (a.south west){}+(0.9cm,0.1cm) to (c);
\end{tikzpicture}
\end{column}
\end{columns}
\end{frame}
%===========================================================================================================
%===========================================================================================================
\section{Search for \BKnn}
\begin{frame}{Outline}
 \tableofcontents[currentsection]
\end{frame}
%===========================================================================================================
\subsection{Motivation}
%===========================================================================================================
\begin{frame}{Lowest-order SM quark level diagrams for $b\to s\nunub$}
\begin{columns}
\begin{column}{0.5\linewidth} 
\bi
\item Loop
\includegraphics[width=0.9\linewidth]{figs/b2snn_loop}
\vspace{0.5cm}
\ei
\end{column}
\begin{column}{0.5\linewidth}
\bi
\item Box \hfill
\includegraphics[width=0.9\linewidth]{figs/b2snn_box}
\vspace{0.5cm}
\ei
\end{column}
\end{columns}
\bi
\item Advantage of \nulnulb (over \ellell): no virtual photon contribution, thus cleaner theoretical prediction.
\ei
\end{frame}
%===========================================================================================================
\begin{frame}{Branching ratio (BR) in the Standard Model and beyond}
\bi
\item Standard Model (SM) prediction: 
\bi
\itemi {$\mathrm{BR}(\Bp\to\Kp\nunub)_{\mathrm{SM}}=(4.6\pm0.5)\times10^{-6}$ \hfill \href{https://arxiv.org/abs/1606.00916}{\color{blue} [1606.00916]}.}
\ei
\itemi Multiple models beyond the SM constrained by $\mathrm{BR}(\Bp\to\Kp\nunub)$:
\bi
\itemi leptoquarks \hfill \href{https://arxiv.org/pdf/1806.05689.pdf}{\color{blue} [1806.05689]}.
\itemi axions \hfill \href{https://arxiv.org/pdf/2002.04623.pdf}{\color{blue} [2002.04623]}.
\itemi dark matter particles \hfill \href{https://arxiv.org/pdf/1911.03490.pdf}{\color{blue} [1911.03490]}.
\itemi ... 
\ei
\ei
\end{frame}
%===========================================================================================================
\begin{frame}{Signal and background}
\bi
\item Signal.
\bi
\itemii {$\epem\to\Y4S\to B^+(\to K^+\nunub)B^-$}.
\ei
\itemii Background.
\bi
\itemii Generic $B$-meson decays: $\epem\to\Y4S\to\BpBm$ or $\BzBzb$.
\itemii Continuum events: $\epem\to q\bar{q}$ or $\tau^{+}\tau^{-}$ ($q=u,d,s,c$ quarks). 
\ei
\ei
\vspace{0.5cm}
\centering
\includegraphics[width=0.5\linewidth]{figs/sphericity_v4}
\end{frame}
%===========================================================================================================
\subsection{Classifiers}
%===========================================================================================================
\begin{frame}{Features}
\bi
\item $\mathcal{O}(100)$ discriminative features considered.
\itemii Feature selection (parallelised):
\bi
\itemiii Iteratively remove variables from the training.
\itemiii Check performance degradation.
\ei
\itemii After selection, boosted decision trees (BDT) trained with 51 features.
\ei

\vspace{0.5cm}

\begin{columns}
\begin{column}{0.333\linewidth}
\centering
\includegraphics[width=1.0\linewidth]{figs/B_sig_weMissM2_ipMask_0}
\end{column}
\begin{column}{0.333\linewidth}
\centering
\includegraphics[width=1.0\linewidth]{figs/sphericity}
\end{column}
\begin{column}{0.333\linewidth}
\centering
\includegraphics[width=1.0\linewidth]{figs/nTracksCleaned}
\end{column}
\end{columns}

\end{frame}
%===========================================================================================================
\begin{frame}{Training}
\bi
\item Manual boosting.
\bi
\itemii Train \bdto on $\mathcal{O}(10^7)$ simulated events.
\itemii Train \bdtt on $\mathcal{O}(10^7)$ simulated events with \bdto$>0.9$.
\ei
\ei
\vspace{0.5cm}
\begin{columns}
\begin{column}{0.5\linewidth}
\includegraphics[width=1.0\linewidth]{figs/significance_bdt1_preliminary}
\end{column}
\begin{column}{0.5\linewidth}
\includegraphics[width=1.0\linewidth]{figs/significance_bdt2_preliminary}
\end{column}
\end{columns}
\end{frame}
%===========================================================================================================
\begin{frame}{Overfitting check}
\bi
\item A bit of overfitting is not problematic as long as the classification behaves the same on simulation and on data.
\bi
\itemiii See F. Dattola's presentation [T81.3].
\ei
\ei
\vspace{0.5cm}
\begin{columns}
\begin{column}{0.5\linewidth}
\includegraphics[width=1.0\linewidth]{figs/overfitting_bdt1_preliminary}
\end{column}
\begin{column}{0.5\linewidth}
\includegraphics[width=1.0\linewidth]{figs/overfitting_bdt2_preliminary}
\end{column}
\end{columns}
\end{frame}
%===========================================================================================================
\begin{frame}{Simulation reweighting using a classifier \hfill {\tiny 
\href{https://iopscience.iop.org/article/10.1088/1742-6596/368/1/012028}{\color{blue!40!gray}[doi:10.1088/1742-6596/368/1/012028]}
}}
\bi
\item Data-driven method to correct mismodeling in simulation.
\begin{enumerate}
	\itemii Train a binary classifier (\bdtoff) to distinguish {\color{blue}simulation} \textit{vs} {\color{red}data}.
\itemii Given the output $p$ of \bdtoff, the simulated events are weighted according to $p/(1-p)$.
\end{enumerate}
\ei
\vspace{0.5cm}
\centering
\includegraphics[width=0.5\textwidth]{figs/overfitting_bdtc_preliminary}
\end{frame}
%===========================================================================================================
\begin{frame}{Simulation reweighting using a classifier \hfill {\tiny 
\href{https://iopscience.iop.org/article/10.1088/1742-6596/368/1/012028}{\color{blue!40!gray}[doi:10.1088/1742-6596/368/1/012028]}
}}
\bi
\item This reweighting procedure improves the data-simulation agreement for data collected below the \Y4S threshold.
\ei
\vspace{0.5cm}
\begin{columns}
\begin{column}{0.5\linewidth}
{\small Before reweighting:}
\includegraphics[width=1.0\linewidth]{figs/B_sig_roeDeltae_ipMask}
\end{column}
\begin{column}{0.5\linewidth}
{\small After reweighting:}
\includegraphics[width=1.0\linewidth]{figs/B_sig_roeDeltae_ipMask_w}
\end{column}
\end{columns}
\end{frame}
%===========================================================================================================
\section{Conclusion and outlook}
%==========================================================================================================
\begin{frame}{Conclusion and outlook}
\bi
\item BDTs are trained to select \BKnn at Belle II.
\itemi A binary classifier can be used to correct simulation mismodeling.
\itemi Other algorithms, such as neural networks, are currently studied.	
\ei
\end{frame}
%===========================================================================================================
\section*{}
%==========================================================================================================
\begin{frame}{}
\centering
\Large Thank you for your attention.
\end{frame}
%===========================================================================================================
\section*{Backup}
%===========================================================================================================
\begin{frame}[noframenumbering]{Nano-beam scheme (idea from Pantaleo Raimondi)}
\begin{columns}
\begin{column}{0.52\linewidth} 
{\small
\bi
\item Goal: $\beta_y^*=0.3\mm$.
\item Hourglass effect limited if $\sigma^{eff}_z<\beta_y^*$.
\item Half crossing angle:
\bi
\item $\phi_x\approx40\mrad$.
\ei
\item Nominal beam spot parameters:
\bi
\item $\sigma_x\approx10\mum$.
\item $\sigma^{eff}_z=\frac{\sigma_x}{\sin\phi_x}\approx0.25\mm$.
\item $\sigma_y\approx50\nm$.
\ei
\ei
\includegraphics[width=0.9\linewidth]{figs/nano_beam_2} 
}
\end{column}
\begin{column}{0.48\linewidth}
\includegraphics[width=1\linewidth]{figs/nano_beam} 
\small\href{https://docs.belle2.org/record/1212?ln=en}{\color{blue!40!gray} [BELLE2-TALK-CONF-2018-142]}
\href{https://arxiv.org/abs/1809.01958}{\color{blue!40!gray} [1809.01958]}
\end{column}
\end{columns}
\end{frame}
%===========================================================================================================

