%===========================================================================================================
\def\bdto {BDT$_1$\xspace}
\def\bdtt {BDT$_2$\xspace}
\def\ptK {$p_{\mathrm{T}}(K^+)$}

\def\bdtoff {BDT$_{\mathrm{c}}$\xspace}

\def\lumiyfours         {62.9\,\invfb\xspace} % 62.93
\def\lumioff            {9.2\,\invfb\xspace} % 9.22
%===========================================================================================================
\newcommand*{\talk}[4]{{#1}, \emph{#2}, \href{#3}{\color{blue!40!gray} [#4]}.}
\newcommand\itemshape[1]{\setbeamertemplate{itemize item}[#1]\usebeamertemplate{itemize item}}
\begin{frame}{Belle II talks at PHENO 2021}
\bi
{\small
\item {\talk{Christoph Schwanda}{Beauty physics from Belle II}{https://indico.cern.ch/event/982783/contributions/4364768/}{link}}
\itemii {\talk{Soumen Halder}{Results and Prospects of Radiative and Electroweak Penguin Decays at Belle (II)}{https://indico.cern.ch/event/982783/contributions/4365603/}{link}}
\itemii {\talk{G\"uney Polat}{Tau physics prospects at Belle II}{https://indico.cern.ch/event/982783/contributions/4364551/}{link}}
\itemii {\talk{Katharina Dort}{Dark-sector physics at Belle II}{https://indico.cern.ch/event/982783/contributions/4362293/}{link}}
\itemii {\talk{Sebastiano Raiz}{Charmless $B$ decays at Belle II}{https://indico.cern.ch/event/982783/contributions/4364552/}{link}}
\itemii {\talk{Chiara La Licata}{The re-discovery of the decays for the CP violation measurements}{https://indico.cern.ch/event/982783/contributions/4364562/}{link}}
{\setbeamercolor{item}{fg=red}\itemiii {\talk{Cyrille Praz}{Search for the rare electroweak decay \BKnn in the early Belle II dataset}{https://indico.cern.ch/event/982783/contributions/4365605/}{link}}}
}
\ei
\end{frame}
%===========================================================================================================
\begin{frame}{Outline}
 \tableofcontents
\end{frame}
%===========================================================================================================
\section{Theoretical motivation}
%===========================================================================================================
\begin{frame}{Branching fraction in the Standard Model}
\bi
\item \BKnn is suppressed in the SM and has never been observed. 
\ei
\vspace{0.1cm}
\begin{columns}
\begin{column}{0.5\linewidth} 
\centering
\includegraphics[width=0.6\linewidth]{figs/feynman_loop_v2}
\end{column}
\begin{column}{0.5\linewidth}
\centering
\includegraphics[width=0.6\linewidth]{figs/feynman_box_v2}
\end{column}
\end{columns}
\bi
\itemiii {$\mathrm{BR}(\Bp\to\Kp\nunub)_{\mathrm{SM}}=(4.6\pm0.5)\times10^{-6}$ \hfill \href{https://arxiv.org/abs/1606.00916}{\color{blue!40!gray} [1606.00916]}.}
\itemiii 10\% theoretical uncertainty mainly from $B\to K$ form factors.
\itemiii {$B\to K$ form factors from \href{https://arxiv.org/abs/1409.4557}{\color{blue!40!gray} [1409.4557]} used for signal simulation.}
\ei
\vspace{0.1cm}
\centering
\includegraphics[width=0.4\linewidth]{figs/q2_ff.pdf}
\end{frame}
%===========================================================================================================
\begin{frame}{Beyond the Standard Model}
\bi
\item Since $\nu_e$, $\nu_\mu$ and $\nu_\tau$ contribute, \BKnn is sensitive to potential lepton flavour universality violation.
\itemi Complementary probe of BSM physics scenarios proposed to explain anomalies observed in $b\to s\ell^+\ell^-$ transitions \href{https://arxiv.org/pdf/2005.03734.pdf}{\color{blue!40!gray} [2005.03734]}. 

\itemi Multiple models beyond the SM constrained by $\mathrm{BR}(\Bp\to\Kp\nunub)$:
\bi
\itemii dark matter particles \hfill \href{https://arxiv.org/pdf/1911.03490.pdf}{\color{blue!40!gray} [1911.03490]}.
\itemii leptoquarks \hfill \href{https://arxiv.org/pdf/1806.05689.pdf}{\color{blue!40!gray} [1806.05689]}.
\itemii axions \hfill \href{https://arxiv.org/pdf/2002.04623.pdf}{\color{blue!40!gray} [2002.04623]}.
\itemii ... 
\ei
\ei
\end{frame}
%===========================================================================================================
\section{The Belle II experiment}
%===========================================================================================================
\begin{frame}[noframenumbering]{Outline}
 \tableofcontents[currentsection]
\end{frame}
%===========================================================================================================
\begin{frame}{The SuperKEKB accelerator}
\begin{columns}
\begin{column}{0.48\linewidth} 
\bi \small
\item{\epem collider in Tsukuba, Japan.}
\itemi{$\sqrt{s}=10.6\gev=\mathrm{m}(\Y4S)$.}
\itemi{$\mathrm{BR}(\Y4S\to\BB)>96\%$.}
\itemi{$\displaystyle\int_{2019.02.04}^{2021.05.21}\,L\,\mathrm{dt}\approx157\invfb$.}
\itemi{World highest instant. luminosity.}
\bi
\item{$L=2.96\times10^{34}\cms$ achieved in May 2021.}
\ei
\ei
\end{column}
\begin{column}{0.52\linewidth}
\centering
\includegraphics[width=0.75\linewidth]{figs/super_kekb}\\
\vspace{0.1cm}
\includegraphics[width=0.9\linewidth]{figs/luminosity_2}
\end{column}
\end{columns}
\end{frame}
%===========================================================================================================
\begin{frame}{{The Belle II detector}}
\begin{columns}
\begin{column}{0.45\linewidth} 
\begin{itemize} \small
\item Pixel Detector (PXD).
\vspace{0.25cm}
\item Silicon Vertex Detector (SVD).
\vspace{0.25cm}
\item Central Drift Chamber (CDC).
\vspace{0.25cm}
\item Calorimeter (ECL).
\vspace{0.25cm}
\item Aerogel Ring-Imaging Cherenkov (ARICH).
\vspace{0.25cm}
\item Time-Of-Propagation (TOP) counter.
\vspace{0.25cm}
\item $K_L^0$ and $\mu$ detection (KLM).
\end{itemize}
\end{column}
\begin{column}{0.55\linewidth}
\centering
\begin{tikzpicture}
\node (a) {\includegraphics[width=0.9\linewidth]{figs/belle2_with_flags}};
\node (b) [below=0.25cm of a] {\includegraphics[width=0.65\linewidth]{figs/vxd.pdf} };
\node(c)  [above=-0.25cm of b, xshift=-1.5cm]{};
\draw[thick, red,->] (a.south west){}+(0.9cm,0.1cm) to (c);
\end{tikzpicture}
\end{column}
\end{columns}
\end{frame}
%===========================================================================================================
\section{Search for \BKnn decays}
%===========================================================================================================
\begin{frame}[noframenumbering]{Outline}
 \tableofcontents[currentsection]
\end{frame}
%===========================================================================================================
\subsection{Introduction}
%===========================================================================================================
\begin{frame}{Data samples used for this analysis}
\bi
\item  {$63\invfb$ collected at $\sqrt{s}=\mathrm{m}(\Y4S)$ ("on-resonance").} 
\itemi {$9\invfb$ collected at $\sqrt{s}=\mathrm{m}(\Y4S)-60\mev$ ("off-resonance").}
\ei
\end{frame}
%===========================================================================================================
\begin{frame}{Signal and background after high-level trigger}
\bi
\item Signal.
\bi
\itemii {$\epem\to\Y4S\to B^+(\to K^+\nunub)B^-$}.
\ei
\itemii Background.
\bi
\itemii Generic $B$-meson decays: $\epem\to\Y4S\to\BpBm\,$ or $\,\BzBzb$.
\itemii Continuum events: $\epem\to q\bar{q}\,$ or $\,\tau^{+}\tau^{-}$ ($q=u,d,s,c$ quarks). 
\ei
\ei
\vspace{0.5cm}
\centering
\includegraphics[width=0.5\linewidth]{figs/sphericity_v4}
\end{frame}
%===========================================================================================================
\begin{frame}{$B$-meson tagging}
\bi
\item Previous searches used tagged approaches, where the second $B$-meson is reconstructed...
\bi
\itemiii ...in a hadronic decay: $\varepsilon_{\mathrm{sig}}=\mathcal{O}(0.04\%)$ \hfill \href{https://arxiv.org/abs/1303.7465v2}{\color{blue!40!gray} [1303.7465 (Babar)]}.
\itemiii ...in a semileptonic decay: $\varepsilon_{\mathrm{sig}}=\mathcal{O}(0.2\%)$ \hfill \href{https://arxiv.org/abs/1702.03224}{\color{blue!40!gray} [1702.03224 (Belle)]}.
\ei
\ei
\vspace{0.25cm}
\centering
\includegraphics[width=0.6\linewidth]{figs/bsig_btag.pdf}
\bi
\itemi In the following, an inclusive tagging approach is used.
\bi
\itemiii No explicit reconstruction of the second $B$-meson.
\itemiii Exploitation of the distinctive topological features of \BKnn.
\ei
\ei
\end{frame}
%===========================================================================================================
\begin{frame}{Signal kaon candidate selection}
\bi
\item Highest-$p_T$ track in event as $\Kp$ candidate.
\bi
\itemii Correct candidate in 80\% of the cases.
\itemii PID requirement to suppress pion background.
\ei
\ei
\vspace{0.5cm}
\centering
\includegraphics[width=0.5\linewidth]{figs/B_sig_K_pt.pdf}
\end{frame}
%===========================================================================================================
\subsection{Binary classification}
%===========================================================================================================
\begin{frame}[noframenumbering]{Outline}
 \tableofcontents[currentsection, currentsubsection]
\end{frame}
%===========================================================================================================
\begin{frame}{Features}
\bi
\itemii Boosted decision trees (BDT) \href{https://arxiv.org/abs/1609.06119}{\color{blue!40!gray} [1609.06119]} trained with 51 features.
\bi
\itemiii Event topology (Fox-Wolfram moments, sphericity, ...).
\itemiii Rest-of-event (ROE) variables.
\itemiii Missing energy, momentum.
\itemiii Vertex separation.
\itemiii ...
\ei
\ei
\vspace{0.5cm}
\centering
\begin{columns}
\begin{column}{0.333\linewidth}
\centering
\includegraphics[width=1.0\linewidth]{figs/B_sig_roeDeltae_ipMask.pdf}
\end{column}
\begin{column}{0.333\linewidth}
\centering
\includegraphics[width=1.0\linewidth]{figs/B_sig_weMissM2_ipMask_0.pdf}
\end{column}
\begin{column}{0.333\linewidth}
\centering
\vspace{-2.05cm}
\includegraphics[width=1.0\linewidth]{figs/sphericity_v4.pdf}\\
\vspace{0.3cm}
\includegraphics[width=1.0\linewidth]{figs/sphericity.pdf}
\end{column}
\end{columns}
\end{frame}
%===========================================================================================================
\begin{frame}{Binary classifiers in series}
\bi
\item Train \bdto on $\mathcal{O}(10^7)$ simulated events.
\item Train \bdtt on $\mathcal{O}(10^7)$ simulated events with \bdto$>0.9$.
\item {$\max(S/\sqrt{S+B})$ reached around \bdtt$>0.95$.}
\ei
\vspace{0.25cm}
\begin{columns}
\begin{column}{0.5\linewidth}
\centering
\includegraphics[width=0.7\linewidth]{figs/overfitting_bdt1.pdf}
\includegraphics[width=0.7\linewidth]{figs/overfitting_bdt2.pdf}
\end{column}
\begin{column}{0.5\linewidth}
\centering
\includegraphics[width=0.7\linewidth]{figs/significance_bdt1.pdf}
\includegraphics[width=0.7\linewidth]{figs/significance_bdt2.pdf}
\end{column}
\end{columns}
\end{frame}
%===========================================================================================================
\begin{frame}{Validation channel: $\Bp\to\Kp\jpsi(\to\cancel{\mumu})$}
\bi
\item To check the data-simulation agreement, $\Bp\to\Kp\jpsi(\to\mumu)$ decays are selected.
\bi
\itemiii Muons are removed from the reconstruction to mimic the signal.
\itemiii Kaon 3-momentum is sampled from simulated signal events.
\ei
\ei
\vspace{0.25cm}
\centering
\includegraphics[width=0.7\linewidth]{figs/MC_and_data_bdt.pdf}
\end{frame}
%===========================================================================================================
\subsection{Signal-strength extraction}
%===========================================================================================================
\begin{frame}[noframenumbering]{Outline}
 \tableofcontents[currentsection, currentsubsection]
\end{frame}
%===========================================================================================================
\begin{frame}{Statistical model}
\bi
\item Binned likelihood defined in the \bdtt$\times$\ptK$\times\sqrt{s}$ space.
\itemiii {$4\times3\times2=24$ bins.}
\bi
\itemiii {\bdtt$\in[0.93,\,0.95,\,0.97,\,0.99,\,1.00]$.}
\itemiii {\ptK$\in[0.5,\,2.0,\,2.4,\,3.5]\gevc$.}
\itemiii {$\sqrt{s}\in\{\mathrm{m}(\Y4S),\, \mathrm{m}(\Y4S)-60\mevcc\}$.}
\ei
\ei
\vspace{0.5cm}
\centering
\includegraphics[width=0.5\linewidth]{figs/bins_2D.pdf}
\end{frame}
%===========================================================================================================
\begin{frame}{Statistical model}
\bi
\item Likelihood function = product of Poisson probability density functions combining the information from the 24 bins. 
\bi
\itemii Templates for the yields of the signal and background derived from simulation.
\itemii Implementation in the \href{https://github.com/scikit-hep/pyhf}{\color{blue!40!gray} pyhf} package, maximum-likelihood fit using \href{https://www.scipy.org/about.html}{\color{blue!40!gray} scipy}.
\ei
\itemi Fit parameters:
\bi
\itemii Signal strength $\mu$ (factor w.r.t. SM expectation for signal yield).
\itemii Nuisance parameters to include the systematic uncertainties \textit{via} event-count modifiers.
\bi
\itemii Main systematic source: background yield normalisation.
\ei
\ei
\ei
\end{frame}
%===========================================================================================================
\begin{frame}{Fit to data}
\bi
\item Maximum-likelihood fit to 24 bins of the \bdtt$\times$\ptK$\times\sqrt{s}$ space.
\ei
\vspace{0.5cm}
\begin{columns}
\begin{column}{0.5\linewidth}
\bi
\item On-resonance.
\ei
\centering
\includegraphics[width=1.0\linewidth]{figs/SRs_postfit.pdf}
\end{column}
\begin{column}{0.5\linewidth}
\bi
\item Off-resonance.
\ei
\centering
\includegraphics[width=1.0\linewidth]{figs/postfit_offresCR.pdf}
\end{column}
\end{columns}\end{frame}
%===========================================================================================================
\begin{frame}{Result}
\bi
\item {\small $\mu = 4.2^{+3.4}_{-3.2} = 4.2^{+2.9}_{-2.8}(\mathrm{stat}){}^{+1.8}_{-1.6}(\mathrm{syst})$.}
\itemii {\small $\mathrm{BR}(\BKnn)=\left[1.9^{+1.6}_{-1.5}\right] \times 10^{-5} = \left[1.9^{+1.3}_{-1.3}(\mathrm{stat}){}^{+0.8}_{-0.7}(\mathrm{syst})\right] \times 10^{-5}$.}
\ei
\begin{columns}
\begin{column}{0.45\linewidth}
\centering
\includegraphics[width=1.0\linewidth]{figs/mu_profile.pdf}
\end{column}
\begin{column}{0.55\linewidth}
\centering
\includegraphics[width=1.0\linewidth]{figs/br_comp.pdf}
\end{column}
\end{columns}
\bi
\item Total uncertainty on $\mu$: profile likelihood scan, fitting the model with fixed values of $\mu$ while keeping the other fit parameters free.
\ei
\end{frame}
%===========================================================================================================
\section{Conclusion and outlook}
%==========================================================================================================
\begin{frame}{Conclusion and outlook}
\bi
\item Search for \BKnn decays with an inclusive tagging approach was performed at Belle II with $(63+9)\invfb$ of data.
\itemi {$\mathrm{BR}(\BKnn)=\left[1.9^{+1.6}_{-1.5}\right] \times 10^{-5}$ \hfill ($<4.1\times 10^{-5}$ @ 90\% C.L.).}
\itemi Pre-print available \href{https://arxiv.org/abs/2104.12624}{\color{blue!40!gray} [2104.12624]}, submitted for publication.
\itemi Next iteration of the analysis will include:
\bi
\itemii More data.
\itemii More channels ($\B^0\to K^{*0}\nu\overline{\nu}$, $\B^0\to K^0_S\nu\overline{\nu}$, ...).
\itemii More classifiers (neural networks).
\ei	
\ei
\end{frame}
%===========================================================================================================
\section*{}
%==========================================================================================================
\begin{frame}{}
\centering
{\Large Thank you for your attention!}\\ 


\end{frame}
%===========================================================================================================
\section*{Backup}
%===========================================================================================================
\begin{frame}[noframenumbering]{Signal kaon candidate selection and event pre-selection}
\begin{columns}
\begin{column}{0.5\linewidth}
\bi
\item Basic track cleanup:
\bi
\itemiii {\footnotesize $p_T>0.1\gevc$, $\theta\in$ CDC, $|dr|<0.5\cm$, $|dz|<3.0\cm$.}
\ei
\itemiii Highest-$p_T$ clean track in event as $\Kp$ candidate.
\bi
\itemiii Correct candidate in 80\% of the cases.
\itemiii \# PXD hits $\ge 1$.
\itemiii PID requirement to suppress pion background.
\ei
\itemiii Loose preselection: 
\bi
\itemiii {$4\leq N_{\mathrm{tracks}}\leq10$.}
\itemiii {$0.3<\theta(\mathbf{p_{miss}})<2.8 \rad$.}
\itemiii {$E_{\mathrm{visible}}>4\gev$.}
\ei
\ei
\end{column}
\begin{column}{0.5\linewidth}
\centering
\includegraphics[width=0.85\linewidth]{figs/nTracksCleaned.pdf}
\includegraphics[width=0.85\linewidth]{figs/B_sig_K_pt.pdf}
\end{column}
\end{columns}
\end{frame}
%===========================================================================================================
\begin{frame}[noframenumbering]{Signal efficiency at \bdtt$>0.95$}
\bi
\itemi At \bdtt$>0.95$, the signal efficiency is 12.5\% for $q^2\approx0$ and drops to zero for $q^2>16\gev^2/c^4$.
\bi
\itemii Sensitive to potential light dark matter candidates.
\ei
\ei
\vspace{0.5cm}
\centering
\includegraphics[width=0.49\linewidth]{figs/q2_eff.pdf}
\includegraphics[width=0.49\linewidth]{figs/q2_ff.pdf}
\end{frame}
%===========================================================================================================
\begin{frame}[noframenumbering]{Background composition at \bdtt $>0.93$}
\bi
\item At \bdtt$>0.93$, $D$-mesons contribute a lot to the remaining  background from $B$-meson decays.
\ei
\vspace{0.5cm}
\begin{columns}
\begin{column}{0.5\linewidth}
\centering
\includegraphics[width=1.0\linewidth]{figs/charged_sig_side_chart.pdf}
\end{column}
\begin{column}{0.5\linewidth}
\centering
\includegraphics[width=1.0\linewidth]{figs/mixed_sig_side_chart.pdf}
\end{column}
\end{columns}
\end{frame}
%===========================================================================================================
\begin{frame}[noframenumbering]{Fit validation}
\bi
\item Toys  generated  for  the simulated data set.
\bi
\itemii Poisson statistical fluctuations.
\itemii Gaussian systematic fluctuations.
\ei
\ei
\vspace{0.25cm}
\begin{columns}
\begin{column}{0.5\linewidth}
\bi
\item {\scriptsize Signal injection study, $\mu_{\mathrm{sig}}\in\{1,\,5,\,20\}$}.
\ei
\centering
\includegraphics[width=1.0\linewidth]{figs/signal_pulls_pyhf.pdf}
\end{column}
\begin{column}{0.5\linewidth}
\bi
\item {\scriptsize Data-model compatibility.}
\ei
\centering
\includegraphics[width=1.0\linewidth]{figs/data_model_compatibility_pyhf.pdf}
\end{column}
\end{columns}
\end{frame}
%===========================================================================================================
\begin{frame}[noframenumbering]{Limit setting}
\bi
\item Expected and observed upper limits on the branching fraction are determined using the \href{https://iopscience.iop.org/article/10.1088/0954-3899/28/10/313}{\color{blue!40!gray} CLs method}.
\ei
\centering
\vspace{0.5cm}
\includegraphics[width=0.5\linewidth]{figs/B_Knunu_CLs_unblinding_90CL_both_expected_and_observed.pdf}
\end{frame}
%===========================================================================================================
