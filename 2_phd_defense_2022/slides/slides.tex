%===========================================================================================================
\begin{frame}[noframenumbering]{Outline}
 \tableofcontents[subsubsectionstyle=hide]
\end{frame}
%===========================================================================================================
\section{Theoretical motivation}
%===========================================================================================================
\begin{frame}{Introduction}
\bi
\item The Standard Model (SM) describes the known elementary particles and their interactions.
\itemi Studies of $B$-meson decays are interesting to test the predictions of the SM and potentially detect the presence of new physics effects.
\itemi {The decay \BKnn is a weak process involving a $b\to s$ transition.}
\ei
\end{frame}
%===========================================================================================================
\begin{frame}{\BKnn in the Standard Model}
\bi
\item \BKnn is suppressed in the SM and has never been observed.
\itemii Decay described by an effective Hamiltonian.
\ei

\vspace{0.15cm}
\begin{tikzpicture}
\node (a) {\includegraphics[width=0.3\linewidth]{figs/phd_thesis/theoretical_motivation/feynman/feynman_loop.pdf}};
\node (c) [right=0.1cm of a] {\includegraphics[width=0.3\linewidth]{figs/phd_thesis/theoretical_motivation/feynman/feynman_box.pdf}};
\node (d) [right=0.01cm of c]{$\rightarrow$};
\node (e) [right=0.01cm of d] {\includegraphics[width=0.3\linewidth]{figs/phd_thesis/theoretical_motivation/feynman/feynman_effective_vertex.pdf}};
\end{tikzpicture}
\vspace{-0.2cm}

\bi
\item {$\mathcal{H}_{\mathrm{eff}}^{\mathrm{SM}}=-\frac{4G_F}{\sqrt{2}}\,V_{tb}\,V_{ts}^*\,C_L^{\mathrm{SM}}\,\mathcal{O}_L+\mathrm{h.c.}$}
\bi
\itemii {$G_F$: Fermi constant.}
\itemii {$V_{ij}$: Elements of the Cabibbo-Kobayashi-Maskawa matrix.}
\itemii {$\mathcal{O}_{L}$: effective vertex operator, $\mathcal{O}_{L}=\frac{e^2}{16\pi^2}\,(\bar{s}_L\gamma_{\mu}b_L)\,(\bar{\nu}_L\gamma^\mu\nu_L)$.}
\itemii {$C_L^{\mathrm{SM}}$: effective coupling constant (Wilson coefficient).}
\ei
\ei

\end{frame}
%===========================================================================================================
\begin{frame}{Branching fraction in the Standard Model}

\bi
\item {$ \frac{\dd\text{\Br}(\BKnn)_\text{SM}}{\dd q^2}
=3\,
\tau_{B}\,
\left|
\frac{G_F \alpha}{16\pi^2}\,
\sqrt{\frac{m_B^3}{3\pi}}\,
V_{tb}V_{ts}^*\,
C_L^{\mathrm{SM}}\,
f_+(q^2)\,
\right|^2
\left(\frac{\lambda_K(q^2)}{m_B^4}\right)^{\frac{3}{2}}$}
\bi
\itemi {$q^2$: Squared invariant mass of the two-neutrino system.}
\itemi {$\tau_B,\,m_B$: Lifetime and mass of the $B$ meson.}
\itemi {$\alpha$: Electromagnetic coupling.}
\itemi {$f_+(q^2)$: Hadronic form factor ($q^2$-dependence of $\left\langle K\right|\bar{s}\gamma_{\mu}b\left|B\right\rangle$).}
\itemi {$\lambda_K(q^2)$: Phase-space factor.}
\ei
\ei

\end{frame}
%===========================================================================================================
\begin{frame}{Branching fraction in the Standard Model}

\bi
\item {$\Br(\BKpnn)_{\mathrm{SM}}=\mypow{(4.6\pm0.5)}{-6}$. \hfill \href{https://doi.org/10.1016/j.ppnp.2016.10.001}{\color{blue!40!gray} \tiny [{Prog.~Part.~Nucl.~Phys.~\textbf{92}, 50 (2017)]}}}
\itemii {$\Br(\BKzznn)_{\mathrm{SM}}=\mypow{(4.3\pm0.5)}{-6}$.}
\itemii 10\% theoretical uncertainty mainly from $B\to K$ form factor.
\itemii {$B\to K$ form factor used for signal simulation. \hfill \href{https://doi.org/10.1007/JHEP02(2015)184}{\color{blue!40!gray}\tiny [J.~High Energy Phys.~\textbf{02}, 184 (2015)]}}
\ei
\vspace{0.25cm}
\centering
\includegraphics[width=0.5\textwidth]{figs/phd_thesis/theoretical_motivation/phase_space_vs_ff.pdf}

\end{frame}
%===========================================================================================================
\begin{frame}{New Physics (NP)}
\bi
\item Since $\nu_e$, $\nu_\mu$ and $\nu_\tau$ contribute, \BKnn is sensitive to potential lepton flavor universality violation.
\itemii Complementary probe of NP scenarios proposed to explain anomalies observed in $b\to s\ell^+\ell^-$ transitions. \hfill \href{https://doi.org/10.1016/j.physletb.2020.135769}{\color{blue!40!gray} \tiny [Phys.~Lett.~B \textbf{809}, 135769 (2020)]}
\itemii Multiple models beyond the SM constrained by $\mathrm{BR}(\BKnn)$:
\bi
\itemiii axions (dark matter candidate). \hfill \href{https://doi.org/10.1103/PhysRevD.102.015023}{\color{blue!40!gray} \tiny [Phys.~Rev.~D \textbf{102}, 015023 (2020)]}
\itemiii leptoquarks. \hfill \href{https://doi.org/10.1103/PhysRevD.98.055003}{\color{blue!40!gray} \tiny [Phys.~Rev.~D \textbf{98}, 055003 (2018)]}
\ei
\ei
\vspace{0.2cm}
\centering
\includegraphics[width=0.4\textwidth]{figs/phd_thesis/theoretical_motivation/feynman/feynman_leptoquark.pdf}
\end{frame}
%===========================================================================================================
\section{The Belle II experiment}
%===========================================================================================================
\begin{frame}[noframenumbering]{Outline}
 \tableofcontents[currentsection, subsubsectionstyle=hide]
\end{frame}
%===========================================================================================================
\begin{frame}{The SuperKEKB accelerator}
\begin{columns}
\begin{column}{0.48\linewidth} 
\bi \small
\item{\epem collider in Tsukuba, Japan.}
\itemi{$\sqrt{s}=10.6\gev=\mathrm{m}(\Y4S)$.}
\itemi{$\mathrm{BR}(\Y4S\to\BB)>96\%$.}
\itemi{$\displaystyle\int_{25.03.2019}^{22.06.2022}\,L\,\mathrm{dt}\approx363\invfb$.}
\bi
\itemii {This analysis: $189/63 \invfb$.}
\ei
\itemi{Long-term target:}
\bi
\itemii {$\displaystyle\int\,L\,\mathrm{dt}= 50\invab$.}
\ei
\ei
\end{column}
\begin{column}{0.52\linewidth}
\centering
\includegraphics[width=1.0\linewidth]{figs/phd_thesis/experimental_setup/superkekb}\\
\end{column}
\end{columns}
\end{frame}
%===========================================================================================================
\begin{frame}{Main \epem processes at $\sqrt{s}=10.6\gev$}
\centering
\adjustbox{max height=1.0\textheight,max width=1.0\textwidth}{
\begin{tabular}{ll}
Process & Cross-section [$\mathrm{nb}$] \\
\midrule
$\epem\to\Y4S$ & 1.10 \\
\midrule
$\epem\to\uubar(\gamma)$ & 1.61 \\
$\epem\to\ddbar(\gamma)$ & 0.40 \\
$\epem\to\ssbar(\gamma)$ & 0.38 \\
$\epem\to\ccbar(\gamma)$ & 1.30 \\
$\epem\to\tautau(\gamma)$ & 0.92 \\
\midrule
$\epem\to\epem(\gamma)$ & 300 \\
$\epem\to\epem\epem$ & 39.7 \\
$\epem\to\epem\mumu$ & 18.9 \\
$\epem\to\gamma\gamma(\gamma)$ & 4.99 \\
$\epem\to\mumu(\gamma)$ & 1.15 \\
\bottomrule

\end{tabular}
}
\end{frame}
%===========================================================================================================
\begin{frame}{{The Belle II detector}}
\begin{columns}
\begin{column}{0.45\linewidth} 
\begin{itemize} \small
\item Pixel Detector (PXD).
\vspace{0.25cm}
\item Silicon Vertex Detector (SVD).
\vspace{0.25cm}
\item Central Drift Chamber (CDC).
\vspace{0.25cm}
\item Calorimeter (ECL).
\vspace{0.25cm}
\item Aerogel Ring-Imaging Cherenkov (ARICH).
\vspace{0.25cm}
\item Time-Of-Propagation (TOP) counter.
\vspace{0.25cm}
\item $K_L^0$ and $\mu$ detection (KLM).
\end{itemize}
\end{column}
\begin{column}{0.55\linewidth}
\centering
\begin{tikzpicture}
\node (a) {\includegraphics[width=0.9\linewidth]{figs/phd_thesis/experimental_setup/belleii_new.pdf}};
\node (b) [below=0.25cm of a] {\includegraphics[width=0.65\linewidth]{figs/vxd.pdf} };
\node(c)  [above=-0.25cm of b, xshift=-1.5cm]{};
\draw[thick, red,->] (a.south west){}+(0.9cm,0.1cm) to (c);
\end{tikzpicture}
\end{column}
\end{columns}
\end{frame}
%===========================================================================================================
\section{Search for \BKpnn decays}
%===========================================================================================================
\begin{frame}[noframenumbering]{Outline}
 \tableofcontents[currentsection, subsubsectionstyle=hide]
\end{frame}
%===========================================================================================================
\subsection{Introduction}
%===========================================================================================================
\begin{frame}{$B$-meson tagging}
\bi
\item Previous searches used tagging methods, where the second $B$-meson ($B_{\mathrm{tag}}$) is reconstructed...
\bi
\itemiii ...in a hadronic decay: $\varepsilon_{\mathrm{sig}}=\mathcal{O}(0.04\%)$. \hfill \href{https://doi.org/10.1103/PhysRevD.87.112005}{\color{blue!40!gray} \tiny [Phys.~Rev.~D \textbf{87}, 112005 (2013) (Babar)]}
\itemiii ...in a semileptonic decay: $\varepsilon_{\mathrm{sig}}=\mathcal{O}(0.2\%)$. \hfill \href{https://doi.org/10.1103/PhysRevD.96.091101}{\color{blue!40!gray} \tiny [Phys.~Rev.~D \textbf{96}, 091101 (2017) (Belle)]}
\ei
\ei
\vspace{0.25cm}
\centering
\includegraphics[width=0.6\linewidth]{figs/bsig_btag.pdf}
\bi
\itemi In the following, an inclusive tagging method is used.
\bi
\itemiii No explicit reconstruction of the second $B$-meson.
\itemiii Exploitation of distinctive topological properties of \BKpnn.
\ei
\ei
\end{frame}
%===========================================================================================================
\begin{frame}{Reconstructed invariant mass of the two-neutrino system}
\bi
\item Invariant mass of the two-neutrino system not accessible.
\bi
\itemii {$q^2=(\mathbf{P}^*_B-\mathbf{P}^*_K)^2=m_B^2+m_K^2-2\,E^*_BE^*_K+2\,\mathbf{p}^*_B\cdot\mathbf{p}^*_K$.}
\ei
\itemii Define a reconstructed invariant mass (approximation: $\mathbf{p}^*_B=\mathbf{0}$):
\bi
\itemii {$q_{\mathrm{rec}}^2=m_B^2+m_K^2-2\,m_BE^*_K$.}
\ei
\ei
\vspace{0.5cm}
\centering
\includegraphics[width=0.4\linewidth]{figs/phd_thesis/search_for_b2hnn/q2rec_vs_q2gen/Bplus2Kplus_Bsig_H_reconstructed_q2_vs_Q2_gen.pdf}
\end{frame}
%===========================================================================================================
\begin{frame}{Signal candidate selection and event pre-selection}
\begin{columns}
\begin{column}{0.5\linewidth}
\bi
\item $\Bp$ candidate with smallest \qrec selected as signal candidate.
\bi
\itemiii Correct candidate in $>90\%$ of the cases.
\itemiii PID requirement to suppress pion background.
\ei
\itemiii Loose event pre-selection: 
\bi
\itemiii {$4\leq N_{\mathrm{tracks}}\leq10$.}
\itemiii {$\theta(\mathbf{p_{miss}})\in$ CDC acceptance.}
\itemiii {$E_{\mathrm{visible}}>4\gev$.}
\itemiii {$\qrec>-1\gevcccc$.}
\ei
\ei
\end{column}
\begin{column}{0.5\linewidth}
\centering
\includegraphics[width=0.85\linewidth]{figs/phd_thesis/search_for_b2hnn/data_mc/overlays_Bplus2Kplus_v34_Y4S_BDT1_training/Bsig_H_reconstructed_q2.pdf}
\includegraphics[width=0.85\linewidth]{figs/phd_thesis/search_for_b2hnn/data_mc/overlays_Bplus2Kplus_v34_Y4S_BDT1_training/nTracksCleaned.pdf}
\end{column}
\end{columns}
\end{frame}
%===========================================================================================================
\subsection{Binary classification}
%===========================================================================================================
\begin{frame}[noframenumbering]{Outline}
 \tableofcontents[currentsection, currentsubsection, subsubsectionstyle=hide]
\end{frame}
%===========================================================================================================
\begin{frame}{Binary classification with boosted decision trees}
\bi
\item Selecting \BKpnn decays is challenging.
\bi
\itemiii Small branching fraction.
\itemiii Neutrinos not detected.
\ei
\itemii Boosted decision tree (BDT) for background suppression.
\bi
\itemiii Trained with simulated signal and background events on a set of discriminative variables.
\itemiii Combination of simple decision trees.
\ei
\ei
\vspace{0.25cm}
\centering
\includegraphics[width=0.6\linewidth]{figs/tree/tree.pdf}
\end{frame}
%===========================================================================================================
\begin{frame}{Variables}
\bi
\item Example of discriminative variables:
\bi
\itemiii Rest-of-the-event (ROE) variables.
\itemiii Missing energy, momentum.
\itemiii Event topology.
\ei
\ei
\vspace{0.5cm}
\centering
\begin{columns}
\begin{column}{0.333\linewidth}
\centering
\includegraphics[width=1.0\linewidth]{figs/phd_thesis/search_for_b2hnn/data_mc/overlays_Bplus2Kplus_v34_Y4S_BDT1_training/B_sig_roeDeltae_ipMask.pdf}
$\Delta E_{\mathrm{ROE}}\equiv E_{\mathrm{ROE}}-\frac{\sqrt{s}}{2}$ 
\end{column}
\begin{column}{0.333\linewidth}
\vspace{-0.6cm}
\centering
\includegraphics[width=1.0\linewidth]{figs/phd_thesis/search_for_b2hnn/data_mc/overlays_Bplus2Kplus_v34_Y4S_BDT1_training/B_sig_weMissM2_ipMask_0.pdf}
\end{column}
\begin{column}{0.333\linewidth}
\centering
\vspace{-2.65cm}
\includegraphics[width=1.0\linewidth]{figs/sphericity_v4.pdf}\\
\vspace{0.3cm}
\includegraphics[width=1.0\linewidth]{figs/phd_thesis/search_for_b2hnn/data_mc/overlays_Bplus2Kplus_v34_Y4S_BDT1_training/sphericity.pdf}
\end{column}
\end{columns}
\end{frame}
%===========================================================================================================
\begin{frame}{Event selection with two BDTs in series}
\bi
\item Train \bdto with 12 variables.
\bi
\item {\footnotesize \bdto$>0.9\implies$ signal retention of 85\% / background rejection of 98\%.}
\ei
\item Train \bdtt with 35 variables on simulated events with \bdto$>0.9$.
\ei
\vspace{0.25cm}
\begin{columns}
\begin{column}{0.5\linewidth}
\centering
\includegraphics[width=0.7\linewidth]{figs/phd_thesis/search_for_b2hnn/classification/BDT1_Bplus2Kplus_v34_overfitting_b2logo.pdf}
\includegraphics[width=0.7\linewidth]{figs/phd_thesis/search_for_b2hnn/classification/BDT2_Bplus2Kplus_v34_overfitting_b2logo.pdf}
\end{column}
\begin{column}{0.5\linewidth}
\centering
\includegraphics[width=0.9\linewidth]{figs/phd_thesis/search_for_b2hnn/classification/Bplus2Kplus_v34_significance_efficiency.pdf}
\end{column}
\end{columns}
\end{frame}
%===========================================================================================================
\begin{frame}{Signal search region}
\bi
\item Translate \bdtt selection into signal efficiency quantile:
\bi
\itemiii {$\tilde{\varepsilon}_{\mathrm{sig}}\equiv 1-\varepsilon_{\mathrm{sig}}(\bdtt)$.}
\ei
\itemii Define signal search region as $\tilde{\varepsilon}_{\mathrm{sig}}>0.92$.
\bi
\itemiii 12 bins in the $\tilde{\varepsilon}_{\mathrm{sig}}\times\qrec$ space.
\ei
\ei
\vspace{0.25cm}
\centering
\includegraphics[width=0.49\linewidth]{figs/phd_thesis/search_for_b2hnn/classification/BDT2_Bplus2Kplus_v34_efficiency_fit.pdf}
\includegraphics[width=0.49\linewidth]{figs/phd_thesis/search_for_b2hnn/signal_search_region_Bplus2Kplus.pdf}
\end{frame}
%===========================================================================================================
\begin{frame}{Expected yields in the signal search region (189\invfb)}
\bi
\item In signal search region, $\epem\to\BpBm$ background is dominant.
\bi
\itemii Background decay contributing the most: $\B\to D^{(*)}\ell\nu$, with $\ell=e,\mu$.
\ei
\ei
\vspace{0.5cm}
\centering
\includegraphics[width=0.49\linewidth]{figs/phd_thesis/search_for_b2hnn/data_mc/overlays_Bplus2Kplus_v34_Y4S_with_continuum_weights/BDT2_Bplus2Kplus_v34_signal_inefficiency_zoom.pdf}
\includegraphics[width=0.49\linewidth]{figs/phd_thesis/search_for_b2hnn/data_mc/overlays_Bplus2Kplus_v34_Y4S_with_continuum_weights/Bsig_H_reconstructed_q2_vs_BDT2_Bplus2Kplus_v34_signal_inefficiency.pdf}
\end{frame}
%===========================================================================================================
\subsection{Comparison between data and simulation}
%===========================================================================================================
\begin{frame}[noframenumbering]{Outline}
 \tableofcontents[currentsection, currentsubsection, subsubsectionstyle=show]
\end{frame}
%===========================================================================================================
\begin{frame}{Data samples}
\bi
\item Data samples used for the comparison between data and simulation:
\bi
\itemi  {$189\invfb$ collected at $\sqrt{s}=\mathrm{m}(\Y4S)$ ("on-resonance").} 
\itemi {$18\invfb$ collected at $\sqrt{s}=\mathrm{m}(\Y4S)-60\mev$ ("off-resonance").}
\ei
\ei
\end{frame}
\subsubsection{Validation channel}
%===========================================================================================================
\begin{frame}{Validation channel: $\Bp\to\Kp\jpsi(\to\mumu)$}
\begin{enumerate}
\item Select $\Bp\to\Kp\jpsi(\to\mumu)$ decays in data and simulation.
\item Remove the muons from the event and replace the kaon by a kaon sampled from simulated signal events.
\item Recalculate the classifier input variables for the modified events.
\item Examine the output of \bdto and \bdtt.
\end{enumerate}
\vspace{0.25cm}
\centering
\includegraphics[width=0.6\linewidth]{figs/phd_thesis/search_for_b2hnn/data_mc/jpsi_validation/Bplus2Kplus_v34_jpsi_validation.pdf}
\end{frame}
%===========================================================================================================
\subsubsection{Off-resonance data and simulation}
%===========================================================================================================
\begin{frame}{Off-resonance data and simulation}
\bi
\item Off-resonance data shape well modelled in the signal search region.
\itemii Normalisation discrepancy factor $\frac{\mathrm{data}}{\mathrm{simulation}}=1.5$ is observed.
\bi
\itemii Corrected on the figure, included as a systematic uncertainty.
\ei
\ei
\vspace{0.5cm}
\centering
\includegraphics[width=0.49\linewidth]{figs/phd_thesis/search_for_b2hnn/data_mc/overlays_Bplus2Kplus_v34_off_resonance_with_continuum_weights/Bsig_H_reconstructed_q2_vs_BDT2_Bplus2Kplus_v34_signal_inefficiency.pdf}
\end{frame}
%===========================================================================================================
\subsubsection{On-resonance data and simulation}
%===========================================================================================================
\begin{frame}{On-resonance data and simulation}
\bi
\item On-resonance data and simulation in the $0.75<\tilde{\varepsilon}_{\mathrm{sig}}<0.90$ sideband.
\itemii Continuum simulation corrected for the normalisation discrepancy observed with off-resonance data.
\ei
\vspace{0.5cm}
\includegraphics[width=0.49\linewidth]{figs/phd_thesis/search_for_b2hnn/data_mc/overlays_Bplus2Kplus_v34_Y4S_sideband_with_scaled_continuum/Bsig_H_reconstructed_q2.pdf}
\includegraphics[width=0.49\linewidth]{figs/phd_thesis/search_for_b2hnn/data_mc/overlays_Bplus2Kplus_v34_Y4S_sideband_with_scaled_continuum/BDT2_Bplus2Kplus_v34_signal_inefficiency.pdf}
\end{frame}
%===========================================================================================================
\subsection{Result obtained with $(63+9)\invfb$ of data}
%===========================================================================================================
\begin{frame}[noframenumbering]{Outline}
 \tableofcontents[currentsection, currentsubsection, subsubsectionstyle=hide]
\end{frame}
%===========================================================================================================
\begin{frame}{Data samples}
\bi
\item Data samples used in the following:
\bi
\itemi  {$63\invfb$ collected at $\sqrt{s}=\mathrm{m}(\Y4S)$ ("on-resonance").} 
\itemi {$9\invfb$ collected at $\sqrt{s}=\mathrm{m}(\Y4S)-60\mev$ ("off-resonance").}
\ei
\itemi Result published in {\href{https://doi.org/10.1103/PhysRevLett.127.181802}{\color{blue!40!gray} Phys.~Rev.~Lett.~127 (2021) 181802}.}
\ei
\end{frame}
%===========================================================================================================
\begin{frame}{Sources of systematic uncertainty}
\bi
\item Physics modelling:
\bi
\itemii {Normalisation of each background source (50\%).}
\itemii {Uncertainty on the branching fractions of $B$ meson decays.}
\itemii {Uncertainty on the $B\to K$ form factor in signal simulation.}
\ei
\itemii Detector modelling:
\bi
\itemii {Modelling of the track-finding efficiency.}
\itemii {Modelling of the measured energy of neutral particles.}
\itemii {Modelling of the PID selection efficiency.}
\ei
\ei
\end{frame}
%===========================================================================================================
\begin{frame}{Expected upper limit for $(63+9)\invfb$ of data}
\bi
\item Examine how the expected upper limit on $\Br(\BKpnn)$ increases when including sources of systematic uncertainties:
\ei
\vspace{0.5cm}
\centering
\includegraphics[width=0.6\linewidth]{figs/br_stat_syst.pdf}
\end{frame}
%===========================================================================================================
\begin{frame}{Statistical model}
\bi
\item Binned likelihood defined in the \bdtt$\times$\ptK$\times\sqrt{s}$ space.
\begin{center}
\includegraphics[width=0.5\linewidth]{figs/bins_2D.pdf}
\end{center}
\vspace{-0.2cm}
\item {$
\mathcal{L}(\mu,\boldsymbol{\theta}|n_1,\,...,\,n_{24})=
\frac{1}{Z}
\prod^{24}_{b=1}\mathrm{Pois}(n_b|\nu_b(\mu,\boldsymbol{\theta}))\,
p(\boldsymbol{\theta})
$.}
\bi
\itemiii {$n_1,...,n_{24}$ ($\nu_1,...,\nu_{24}$) events are observed (expected).}
\itemiii Signal strength $\mu=\frac{\Br(\BKpnn)}{\Br(\BKpnn)_{\mathrm{SM}}}$
\itemiii Nuisance parameters $\boldsymbol{\theta}$ to include systematic uncertainties \textit{via} event-count modifiers.
\ei
\ei
\end{frame}
%===========================================================================================================
\begin{frame}{Result of the fit to $(63+9)\invfb$ of data}
\bi
\item {\small $\mu = 4.2^{+3.4}_{-3.2} = 4.2^{+2.9}_{-2.8}(\mathrm{stat}){}^{+1.8}_{-1.6}(\mathrm{syst})$.}
\itemii {\small $\mathrm{BR}(\BKpnn)=\left[1.9^{+1.6}_{-1.5}\right] \times 10^{-5} = \left[1.9^{+1.3}_{-1.3}(\mathrm{stat}){}^{+0.8}_{-0.7}(\mathrm{syst})\right] \times 10^{-5}$.}
\ei
\vspace{0.5cm}
\begin{columns}
\begin{column}{0.5\linewidth}
\bi
\item Off-resonance.
\ei
\centering
\includegraphics[width=1.0\linewidth]{figs/postfit_offresCR.pdf}
\end{column}
\begin{column}{0.5\linewidth}
\bi
\item On-resonance.
\ei
\centering
\includegraphics[width=1.0\linewidth]{figs/SRs_postfit.pdf}
\end{column}
\end{columns}
\end{frame}
%===========================================================================================================
\begin{frame}{Comparison with previous results}

\centering
\includegraphics[width=0.55\linewidth]{figs/phd_thesis/search_for_b2hnn/first_iteration/br_comp.pdf}

\bi
\item Assuming uncertainty scaling as $1/\sqrt{L}$, inclusive tagging is better than hadronic and semileptonic tagging methods.
\ei

\vspace{0.2cm}

\adjustbox{max height=0.4\textheight,max width=1.0\textwidth}{
\begin{tabular}{llllll}
Experiment & $L$ [\invfb] & Method & $\sigma_{\mathrm{\Br}}$ [$10^{-5}$]& $\sigma_{\mathrm{\Br}}\cdot\sqrt{L}$ [$10^{-4}\sqrt{\invfb}$] & Ref.\\
\hline
Belle & 711 & Hadronic & 1.63 & 4.36 & \href{10.1103/PhysRevD.87.111103}{\color{blue!40!gray} \tiny Phys.~Rev.~D \textbf{87}, 111103 (2013)} \\
Belle & 711 & Semileptonic & 0.57 & 1.51 & \href{https://doi.org/10.1103/PhysRevD.96.091101}{\color{blue!40!gray} \tiny Phys.~Rev.~D \textbf{96}, 091101 (2017)} \\
Babar & 429 & Combined & 0.65 & 1.35 & \href{https://doi.org/10.1103/PhysRevD.87.112005}{\color{blue!40!gray} \tiny Phys.~Rev.~D \textbf{87}, 112005 (2013)} \\
{\color{red} Belle II} & {\color{red} 63} & {\color{red} Inclusive} & {\color{red} 1.55} & {\color{red} 1.23} & \href{https://doi.org/10.1103/PhysRevLett.127.181802}{\color{red} \tiny Phys.~Rev.~Lett.~\textbf{127}, 181802 (2021)} \\ 
\hline

\end{tabular}
}
\end{frame}
%===========================================================================================================
\subsection*{Expected upper limit for 189\invfb of data}
%===========================================================================================================
\begin{frame}{Expected upper limit for 189\invfb of data}
\bi
\item Expected upper limit with 189\invfb of data.
\bi
\itemi {$\Br(\BKpnn)<1.0\times 10^{-5}$ ($90\%\,\mathrm{C.L.}$).}
\ei
\itemi Current best upper limit obtained by Babar with $429\invfb$ of data.
\bi
\itemi {$\Br(\BKpnn)<1.6\times 10^{-5}$ ($90\%\,\mathrm{C.L.}$). \\ \href{https://doi.org/10.1103/PhysRevD.87.112005}{\color{blue!40!gray} \tiny [Phys.~Rev.~D \textbf{87}, 112005 (2013) (Babar)]}} 
\ei
\ei
\end{frame}
%===========================================================================================================
\section{Conclusion and outlook}
%==========================================================================================================
\begin{frame}{Conclusion and outlook}
\bi
\item First search for \BKnn decays with an inclusive tagging method.
\itemii Result published with $(63+9)\invfb$ of Belle II data.
\bi
\itemii {\href{https://doi.org/10.1103/PhysRevLett.127.181802}{\color{blue!40!gray} Phys.~Rev.~Lett.~127 (2021) 181802}.}
\ei
\itemii The inclusive tagging method provides a better sensitivity per integrated luminosity than the tagging methods of previous searches.
\itemii The \textit{expected} upper limit on $\Br(\BKpnn)$ with $(189+18)\invfb$ is better than the current best upper limit.
\itemii The success of the inclusive tagging method opens new opportunities for the search of rare decays with neutrinos in the final state.
\ei
\end{frame}
%===========================================================================================================
\section*{}
%==========================================================================================================
\begin{frame}{}
\centering
{\Large Thank you for your attention.}\\ 


\end{frame}
%===========================================================================================================
\section*{Backup}
%===========================================================================================================
\begin{frame}[noframenumbering]{Bug in background category weighting}
\centering
\includegraphics[width=0.49\linewidth]{figs/BP_Before.pdf}
\includegraphics[width=0.49\linewidth]{figs/BP_After.pdf}
\end{frame}
%===========================================================================================================
\begin{frame}[noframenumbering]{Transition between two generations of quarks}
\bi
\itemiii Transition between two generations of quarks possible because the weak eigenstates are linear combinations of the mass eigenstates:
\ei
\vspace{1.0cm}
\centering
{
$
\begin{pmatrix}
d'\\
s'\\
b'\\
\end{pmatrix}
=
\begin{pmatrix}
V_{ud} & V_{us} & V_{ub} \\
V_{cd} & V_{cs} & V_{cb} \\
V_{td} & V_{ts} & V_{tb} \\
\end{pmatrix}
\begin{pmatrix}
d\\
s\\
b\\
\end{pmatrix}
$
\\
\vspace{1.0cm}
$
\begin{pmatrix}
|V_{ud}| & |V_{us}| & |V_{ub}| \\
|V_{cd}| & |V_{cs}| & |V_{cb}| \\
|V_{td}| & |V_{ts}| & |V_{tb}| \\
\end{pmatrix}
\approx
\begin{pmatrix}
0.974 & 0.226 & 0.004 \\
0.226 & 0.973 & 0.041 \\
0.009 & 0.040 & 0.999 \\
\end{pmatrix}
$
}
\end{frame}
%===========================================================================================================
\begin{frame}[noframenumbering]{Experimental anomalies in $b\to s\ell^+\ell^-$}
\bi
\item {$R_{H}\equiv\frac{\mathrm{\Br}(B\to H\mumu)}{\mathrm{\Br}(B\to H\epem)}$.}
\itemii In 2017, LHCb reported a value of $R_{K^{*0}}$ $2.3\,\sigma$ smaller than the SM prediction.
\itemii In 2020, LHCb reported the result of a fit to the angular variables of $\Bz\to K^{*0}\mumu$ decays showing a $3.3\,\sigma$ deviation from the predicted SM value of a Wilson coefficient specific to \BKll decays called $C_9$.
\itemii In 2021, LHCb reported the result of a fit to the angular variables of $\B^+\to K^{*+}\mumu$ decays showing a $3.1\,\sigma$ deviation from the predicted SM value of the same Wilson coefficient $C_9$.
\itemii In 2022, LHCb reported a value of $R_{K^+}$ $3.1\,\sigma$ smaller than the SM prediction.
\ei
\end{frame}
%===========================================================================================================
\begin{frame}[noframenumbering]{Nano-beam scheme (idea from Pantaleo Raimondi)}
\begin{columns}
\begin{column}{0.52\linewidth} 
{\small
\bi
\item Goal: $\beta_y^*=0.3\mm$.
\item Hourglass effect limited if $\sigma^{eff}_z<\beta_y^*$.
\item Half crossing angle:
\bi
\item $\phi_x\approx40\mrad$.
\ei
\item Nominal beam spot parameters:
\bi
\item $\sigma_x\approx10\mum$.
\item $\sigma^{eff}_z=\frac{\sigma_x}{\sin\phi_x}\approx0.25\mm$.
\item $\sigma_y\approx50\nm$.
\ei
\ei
\includegraphics[width=0.9\linewidth]{figs/nano_beam_2} 
}
\end{column}
\begin{column}{0.48\linewidth}
\includegraphics[width=1\linewidth]{figs/nano_beam} 
\small\href{https://docs.belle2.org/record/1212?ln=en}{\color{blue!40!gray} [BELLE2-TALK-CONF-2018-142]}
\href{https://arxiv.org/abs/1809.01958}{\color{blue!40!gray} [1809.01958]}
\end{column}
\end{columns}
\end{frame}
%===========================================================================================================
\begin{frame}[noframenumbering]{Pixel Detector (PXD)}
\bi
\item Based on depleted p-channel field-effect transistors (DEPFETs).
\itemii 40 modules of $768\times 250$ DEPFET pixels.
\itemii Impact-parameter resolution of approximately 12\mum.
\ei
\vspace{0.5cm}
\centering
\includegraphics[width=0.4\linewidth]{figs/phd_thesis/experimental_setup/pxd_xy}
\end{frame}
%===========================================================================================================
\begin{frame}[noframenumbering]{Silicon Vertex Detector (SVD)}
\bi
\item Based on double-sided silicon strips.
\itemii {$p$-side and $n$-side strips perpendicular to determine coordinates of charged paticles.}
\ei
\vspace{0.25cm}
\centering
\includegraphics[width=0.4\linewidth]{figs/phd_thesis/experimental_setup/svd.pdf}
\includegraphics[width=0.3\linewidth]{figs/phd_thesis/experimental_setup/svd_xy}\\
\includegraphics[width=0.7\linewidth]{figs/phd_thesis/experimental_setup/svd_zx}
\end{frame}
%===========================================================================================================
\begin{frame}[noframenumbering]{Central Drift Chamber (CDC)}
\bi
\item 56 layers of wires with axial or skewed orientation.
\itemii Electron released during ionisation of a gas mixture cause avalanches of electrons.
\itemii {$p_T=|q|B\rho$ with a relative resolution of the order of 0.1\%}.
\ei
\vspace{0.5cm}
\centering
\includegraphics[width=0.4\linewidth]{figs/phd_thesis/experimental_setup/cdc_wires}
\includegraphics[width=0.4\linewidth]{figs/phd_thesis/experimental_setup/cdc}
\end{frame}
%===========================================================================================================
\begin{frame}[noframenumbering]{Particle identification}
\bi
\item Two detectors use Cherenkov radiations for particle identification.
\bi
\itemii Aerogel Ring-Imaging Cherenkov (ARICH).
\itemii Time-Of-Propagation (TOP) counter.
\ei
\itemii Particle identification from $\beta=1/(n \cos \theta_C)$ and $m=p/(\gamma\beta)$.
\ei
\vspace{0.5cm}
\centering
\includegraphics[width=0.9\linewidth]{figs/phd_thesis/experimental_setup/arich_top.pdf}
\end{frame}
%===========================================================================================================
\begin{frame}[noframenumbering]{Example of ROC curve}
\centering
\includegraphics[width=0.49\linewidth]{figs/phd_thesis/data_analysis/bdt_output.pdf}
\includegraphics[width=0.49\linewidth]{figs/phd_thesis/data_analysis/roc_curve.pdf}
\end{frame}
%===========================================================================================================
\begin{frame}[noframenumbering]{Choice of BDT parameters}
\centering
\adjustbox{max height=0.4\textheight,max width=1.0\textwidth}{
\begin{tabular}{llll}
Parameter & Tested values & \bdto & \bdtt \\
\hline
Number of trees & [200, 500, 1000, 2000] & 2000 & 2000 \\
Tree depth & [2, 3, 4, 5, 6] & 2 & 3 \\
Learning rate & [0.05, 0.1, 0.2] & 0.2 & 0.2 \\
Sampling rate & [0.5, 0.8, 1.0] & 0.5 & 0.5 \\
Number of equal-frequency bins & [$2^4$, $2^6$, $2^8$, $2^{10}$, $2^{12}$] & $2^8$ & $2^8$ \\
\hline

\end{tabular}
}
\includegraphics[width=0.49\linewidth]{figs/phd_thesis/search_for_b2hnn/classification/hyperparameter_optimisation_v33_16files_rank.pdf}
\includegraphics[width=0.49\linewidth]{figs/phd_thesis/search_for_b2hnn/classification/hyperparameter_optimisation_v33_16files_ratio.pdf}
\end{frame}
%===========================================================================================================
\begin{frame}[noframenumbering]{Input variables for \bdto}
\centering
\adjustbox{max height=0.4\textheight,max width=1.0\textwidth}{
\begin{tabular}{lll}
Variable & $\BKpnn$ & $\BKznn$ \\
\midrule 
$\Delta E_{\mathrm{ROE}}$ & 0.62682 & 0.59926 \\
Modified Fox-Wolfram $H^{so}_{m,2}$ & 0.08731 & 0.09382 \\
$p_{\mathrm{ROE}}$ & 0.03517  & 0.03869 \\
Modified Fox-Wolfram $H^{so}_{m,4}$ & 0.02372 & 0.01397 \\
Modified Fox-Wolfram $R^{oo}_{0}$ & 0.02003 & 0.00876 \\
Modified Fox-Wolfram $R^{oo}_{2}$ & 0.01988 & 0.02958 \\
$\theta(p_{\mathrm{ROE}})$ & 0.01813 & 0.01584 \\
Harmonic Moment $B_0$ & 0.01732 & 0.02434 \\
$\cos(\mathrm{thrust}_B,\,\mathrm{thrust_{ROE}})$ & 0.01371 & 0.00954 \\
Fox-Wolfram Moment $R_1$ & 0.01072 & 0.03725 \\
$\cos(\theta(\mathrm{thrust}))$ & 0.00953 & 0.00992 \\
Harmonic Moment $B_2$ & 0.00648 & 0.01390 \\
\bottomrule

\end{tabular}
}
\end{frame}
%===========================================================================================================
\begin{frame}[noframenumbering]{Input variables for \bdtt (\BKpnn)}
\centering
\adjustbox{max height=0.4\textheight,max width=1.0\textwidth}{
\begin{tabular}{ll}
Variable & $\Delta$(AUC) \\
\midrule 
Modified Fox-Wolfram $H^{so}_{m,2}$ & $-0.53286$ \\
$\mathrm{cos}(\mathrm{thrust}_B,\,\mathrm{thrust_{ROE}})$ & $-0.11077$ \\
Median(p-value($D^0$)) & $-0.06013$ \\
p-value(Tag Vertex) & $-0.02718$ \\
$M(D^0)$ & $-0.02297$ \\
$\theta(p_{\mathrm{missing}})$ & $-0.02144$ \\
$\Delta E_{\mathrm{ROE}}$ & $-0.01536$ \\
Modified Fox-Wolfram $R^{oo}_{2}$ & $-0.01272$ \\
p-value($D^+$) & $-0.01123$ \\
$N_{\mathrm{lepton}}$ & $-0.00946$ \\
Total charge squared & $-0.00889$ \\
$dz(K^+,\,\mathrm{Tag\,Vertex})$ & $-0.00881$ \\
$dr(K^+)$ & $-0.00688$ \\
$\theta(p_{\mathrm{ROE}})$ & $-0.00641$ \\
Modified Fox-Wolfram $H^{so}_{n,2}$ & $-0.00497$ \\
Modified Fox-Wolfram $H^{so}_{c,2}$ & $-0.00436$ \\
Modified Fox-Wolfram $H^{so}_{m,4}$ & $-0.00409$ \\
$p_{\mathrm{ROE}}$ & $-0.00398$ \\
${M_{\mathrm{missing}}}^2$ & $-0.00354$ \\
Sphericity & $-0.00333$ \\
Fox-Wolfram Moment $R_{2}$ & $-0.00299$ \\
$N_{\mathrm{tracks}}$ & $-0.00263$ \\
$N_{\gamma}$ & $-0.00223$ \\
$dr(D^+)$ & $-0.00220$ \\
\bottomrule 

\end{tabular}
\begin{tabular}{ll}
Variable & $\Delta$(AUC) \\
\midrule 
$dx(\mathrm{Tag\,Vertex})$ & $-0.00212$ \\
p-value($D^0$) & $-0.00198$ \\
$dz(\mathrm{Tag\,Vertex})$ & $-0.00176$ \\
Fox-Wolfram Moment $R_{1}$ & $-0.00126$ \\
$\mathrm{cos}(\mathrm{thrust}_B,\,z)$ & $-0.00125$ \\
$\mathrm{cos(\theta(thrust))}$ & $-0.00121$ \\
Modified Fox-Wolfram $R^{oo}_{0}$ & $-0.00088$ \\
$\mathrm{Variance_{ROE}}(p_T)$ & $-0.00088$ \\
Fox-Wolfram Moment $R_{3}$ & $-0.00058$ \\
$dr(K^+,\,\mathrm{Tag\,Vertex})$ & $-0.00057$ \\
Harmonic Moment $B_0$ & $-0.00045$ \\
\hdashline
$dy(\mathrm{Tag\,Vertex})$ & $-0.00036$ \\
$dr(D^0)$ & $-0.00033$ \\
$M(\mathrm{ROE})$ & $-0.00032$ \\
$\mathrm{thrust_{ROE}}$ & $-0.00019$ \\
$dz(D^0)$ & $-0.00019$ \\
$dz(K^+)$ & $-0.00018$ \\
$dz(D^+)$ & $-0.00012$ \\
$N_{\mathrm{tracks}}+N_{\gamma}$ & $-0.00010$ \\
Harmonic Moment $B_2$ & $-0.00004$ \\
$\phi(K^+)$ & $-0.00003$ \\
Modified Fox-Wolfram $H^{so}_{m,0}$ & $+0.00003$ \\
Thrust & $+0.00013$ \\
\bottomrule 

\end{tabular}
}
\end{frame}
%===========================================================================================================
\begin{frame}[noframenumbering]{Input variables for \bdtt (\BKznn)}
\centering
\adjustbox{max height=0.4\textheight,max width=1.0\textwidth}{
\begin{tabular}{ll}
Variable & $\Delta$(AUC) \\
\midrule 
Modified Fox-Wolfram $H^{so}_{m,2}$ & $-0.68504$ \\
$\mathrm{cos}(\mathrm{thrust}_B,\,\mathrm{thrust_{ROE}})$ & $-0.07418$ \\
$\theta(p_{\mathrm{missing}})$ & $-0.03280$ \\
$\mathrm{cos}(p_{K^0_{\mathrm{S}}},\,\mathrm{line}(\mathrm{IP},\,K^0_\mathrm{S}\,\mathrm{vertex}))$ & $-0.02363$ \\
Fox-Wolfram Moment $R_{2}$ & $-0.01403$ \\
Modified Fox-Wolfram $R^{oo}_{2}$ & $-0.01384$ \\
$\Delta E_{\mathrm{ROE}}$ & $-0.00724$ \\
$N_{\mathrm{lepton}}$ & $-0.00636$ \\
Modified Fox-Wolfram $H^{so}_{m,4}$ & $-0.00613$ \\
$\theta(p_{\mathrm{ROE}})$ & $-0.00612$ \\
$dz(p_{D^0})$ & $-0.00560$ \\
Modified Fox-Wolfram $H^{so}_{c,2}$ & $-0.00503$ \\
$p_{\mathrm{ROE}}$ & $-0.00436$ \\
p-value(Tag Vertex) & $-0.00428$ \\
Modified Fox-Wolfram $H^{so}_{n,2}$ & $-0.00386$ \\
$N_{\mathrm{tracks}}$ & $-0.00360$ \\
$N_{\gamma}$ & $-0.00290$ \\
p-value($D^+$) & $-0.00268$ \\
Harmonic Moment $B_0$ & $-0.00257$ \\
$dz(p_{K^0_\mathrm{S}},\,\mathrm{Tag\,Vertex})$ & $-0.00228$ \\
Total charge squared & $-0.00203$ \\
$dx(\mathrm{Tag\,Vertex})$ & $-0.00188$ \\
Sphericity & $-0.00171$ \\
\bottomrule 

\end{tabular}
\begin{tabular}{ll}
Variable & $\Delta$(AUC) \\
\midrule 
$M(K^0_\mathrm{S})$ & $-0.00137$ \\
$\mathrm{cos}(\mathrm{thrust}_B,\,z)$ & $-0.00135$ \\
${M_{\mathrm{missing}}}^2$ & $-0.00121$ \\
$dr(p_{K^0_{\mathrm{S}}})$ & $-0.00113$ \\
$dz(\mathrm{Tag\,Vertex})$ & $-0.00106$ \\
$\mathrm{cos(\theta(thrust))}$ & $-0.00095$ \\
Fox-Wolfram Moment $R_{3}$ & $-0.00080$ \\
Modified Fox-Wolfram $H^{so}_{m,0}$ & $-0.00071$ \\
$M(D^0)$ & $-0.00068$ \\
$\mathrm{Variance_{ROE}}(p_T)$ & $-0.00061$ \\
$dz(p_{K^0_{\mathrm{S}}})$ & $-0.00054$ \\
Fox-Wolfram Moment $R_{1}$ & $-0.00042$ \\
\hdashline
$M(D^+)$ & $-0.00040$ \\
$dr(p_{K^0_\mathrm{S}},\,\mathrm{Tag\,Vertex})$ & $-0.00034$ \\
Modified Fox-Wolfram $R^{oo}_{0}$ & $-0.00030$ \\
$M(\mathrm{ROE})$ & $-0.00022$ \\
$N_{\mathrm{tracks}}+N_{\gamma}$ & $-0.00021$ \\
$dr(p_{D^0})$ & $-0.00019$ \\
$dy(\mathrm{Tag\,Vertex})$ & $-0.00018$ \\
$dz(p_{D^+})$ & $-0.00011$ \\
Thrust & $-0.00007$ \\
Harmonic Moment $B_2$ & $-0.00002$ \\
$\mathrm{thrust_{ROE}}$ & $+0.00008$ \\
\bottomrule 

\end{tabular}
}
\end{frame}
%===========================================================================================================
\begin{frame}[noframenumbering]{Input variable selection}
\centering
\includegraphics[width=0.49\linewidth]{figs/phd_thesis/search_for_b2hnn/classification/variable_selection_Bplus2Kplus_v34.pdf}
\includegraphics[width=0.49\linewidth]{figs/phd_thesis/search_for_b2hnn/classification/variable_selection_Bzero2Kshort_v34.pdf}
\end{frame}
%===========================================================================================================
\begin{frame}[noframenumbering]{Signal selection efficiency in the signal search region}
\bi
\itemi In the signal search region, the signal efficiency is 15\% for $q^2\approx0$ and drops to zero for $q^2>18\gev^2/c^4$.
\bi
\itemii Sensitive to potential light dark matter candidates.
\ei
\ei
\vspace{0.5cm}
\centering
\includegraphics[width=0.49\linewidth]{figs/phd_thesis/search_for_b2hnn/Bplus2Kplus_v34_efficiency_vs_q2.pdf}
\includegraphics[width=0.49\linewidth]{figs/phd_thesis/theoretical_motivation/phase_space_vs_ff.pdf}
\end{frame}
%===========================================================================================================
\begin{frame}[noframenumbering]{Validation channel: $\Bp\to\Kp\jpsi(\to\cancel{\mumu})$}
\begin{enumerate}
\item Select $\Bp\to\Kp\jpsi(\to\mumu)$ decays in data and simulation.
\bi
\itemii {$ \left|M_{\mu\mu}-M^{\mathrm{PDG}}_{\jpsi}\right|<0.05\gevcc $}
\itemii {$ |\Delta E|\equiv\left|E_B^*-\frac{\sqrt{s}}{2}\right|<0.1\gev $}
\itemii {$ M_{\mathrm{bc}}\equiv\sqrt{\left(\frac{\sqrt{s}}{2c^2}\right)^2-\left(\frac{p^*_B}{c}\right)^2}>5.25\gevcc $}
\ei
\end{enumerate}
\vspace{0.25cm}
\centering
\includegraphics[width=0.7\linewidth]{figs/phd_thesis/search_for_b2hnn/data_mc/jpsi_validation/Bplus2Kplus_v34_jpsi_skim.pdf}
\end{frame}
%===========================================================================================================
\begin{frame}[noframenumbering]{Correction of continuum background mis-modelling}
\bi
\item Data-driven method to correct mis-modelling of off-resonance data.
\begin{enumerate}
	\itemii Train a binary classifier (\bdtc) to distinguish {\color{blue}simulation} \textit{vs} {\color{red}data}.
\itemii Given the output of \bdtc, the simulated events are weighted according to $\mathrm{\bdtc}/(1-\mathrm{\bdtc})$.
\end{enumerate}
\ei
\vspace{0.5cm}
\centering
\includegraphics[width=0.495\textwidth]{figs/phd_thesis/search_for_b2hnn/classification/BDTC_Bplus2Kplus_v34_overfitting_b2logo.pdf}
\end{frame}
%===========================================================================================================
\begin{frame}[noframenumbering]{Correction of continuum background mis-modelling}
\bi
\item Agreement between off-resonance data and simulation improved.
\ei
\vspace{0.5cm}
\begin{columns}
\begin{column}{0.5\linewidth}
{\small Before reweighting:}
\includegraphics[width=1.0\linewidth]{figs/phd_thesis/search_for_b2hnn/data_mc/overlays_Bplus2Kplus_v34_off_resonance_no_continuum_weights/B_sig_roeDeltae_ipMask.pdf}
\end{column}
\begin{column}{0.5\linewidth}
{\small After reweighting:}
\includegraphics[width=1.0\linewidth]{figs/phd_thesis/search_for_b2hnn/data_mc/overlays_Bplus2Kplus_v34_off_resonance_with_continuum_weights/B_sig_roeDeltae_ipMask.pdf}
\end{column}
\end{columns}
\end{frame}
%===========================================================================================================
\begin{frame}[noframenumbering]{Statistical model}
\bi
\item {$\{\mathrm{samples}\}=\{\mathrm{Signal},\,\uubar,\,\ddbar,\,\ccbar,\,\ssbar,\,\tau\overline{\tau},\,\BzBzb,\,\BpBm\}$.}
\itemii 24 bins where $n_1,...,n_{24}$ ($\nu_1,...,\nu_{24}$) events are observed (expected).
\itemii Fit parameters:
\bi
\itemii Signal strength $\mu=\Br(\BKnn)/\Br(\BKnn)_{\mathrm{SM}}$.
\itemii Nuisance parameters $\boldsymbol{\theta}=\begin{pmatrix} \mu_{\uubar},\,\mu_{\ddbar},\,...,\,\mu_{\BpBm},\,\theta_{8},\,...,\,\theta_{N} \end{pmatrix}^T$ to include systematic uncertainties \textit{via} event-count modifiers.
\ei
\itemii Likelihood function = product of Poisson probability density functions combining the information from the 24 bins. 
\bi
\itemii {$
\mathcal{L}(\mu,\boldsymbol{\theta}|n_1,\,...,\,n_{24})=
\frac{1}{Z}
\prod^{24}_{b=1}\mathrm{Pois}(n_b|\nu_b(\mu,\boldsymbol{\theta}))\,
p(\boldsymbol{\theta})
$.}
\itemii {$
\nu_{b}(\mu,\boldsymbol{\theta})=
\sum_{s\in\{\mathrm{samples}\}}\mu_s\left(\nu^{0}_{bs}+\Delta_{bs}(\boldsymbol{\theta})\right)
$.}
\ei
\ei
\end{frame}
%===========================================================================================================
\begin{frame}[noframenumbering]{Fit validation}
\bi
\item Toys  generated  for  the simulated data set.
\bi
\itemii Poisson statistical fluctuations.
\itemii Gaussian systematic fluctuations.
\ei
\ei
\vspace{0.25cm}
\begin{columns}
\begin{column}{0.5\linewidth}
\bi
\item {\scriptsize Signal injection study, $\mu_{\mathrm{sig}}\in\{1,\,5,\,20\}$}.
\ei
\centering
\includegraphics[width=1.0\linewidth]{figs/signal_pulls_pyhf.pdf}
\end{column}
\begin{column}{0.5\linewidth}
\bi
\item {\scriptsize Data-model compatibility.}
\ei
\centering
\includegraphics[width=1.0\linewidth]{figs/data_model_compatibility_pyhf.pdf}
\end{column}
\end{columns}
\end{frame}
%===========================================================================================================
\begin{frame}[noframenumbering]{Upper limit determination with the \CLs method}
\bi
\item Likelihood ratio: $
\lambda(\mu)=
\frac{\mathcal{L}(\mu,\boldsymbol{\hat{\hat{\theta}}}\,|\,n_1,\,...,\,n_{N_b})}
{\mathcal{L}(\hat{\mu},\boldsymbol{\hat{\theta}}\,|\,n_1,\,...,\,n_{N_b})}$.
\itemii Test statistic: $
q_\mu=
\begin{cases}
-2\ln\lambda(\mu) & \text{if } \mu\ge\hat{\mu}, \\
0 & \text{otherwise}.
\end{cases}
$
\itemii {$p$-value (level of agreement between data and hypothesised $\mu$):}
\bi
\itemii {$p_{s+b}=P(q_\mu>q_{\mu,\mathrm{obs}}\,|\,\mu)=\int_{q_{\mu,\mathrm{obs}}}^{\infty}p(q_\mu|\mu)\,\dd q_{\mu}$.}
\ei
\itemii {$p$-value (background-only hypothesis):}
\bi
\itemii {$p_{b}=P(q_\mu>q_{\mu,\mathrm{obs}}\,|\,0)=\int_{q_{\mu,\mathrm{obs}}}^{\infty}p(q_\mu|0)\,\dd q_{\mu}$.}
\ei
\itemii {\CLs ratio: $\CLs=\frac{p_{s+b}}{p_{b}}$.}
\itemii {$90\%$ CL upper limit on $\mu$: largest value of $\mu$ such that $\CLs\ge0.1$.}
\ei
\end{frame}
%===========================================================================================================
\begin{frame}[noframenumbering]{Upper limit on $\Br(\BKpnn)$ with $(63+9)\invfb$ of data}
\bi
\item Upper on the branching fraction determined using the \CLs method.
\itemi {$\Br(\BKpnn)<4.1\times 10^{-5}\,@\,90\% \mathrm{C.L.}$}
\ei
\vspace{0.5cm}
\centering
\includegraphics[width=0.5\linewidth]{figs/B_Knunu_CLs_unblinding_90CL_both_expected_and_observed.pdf}
\end{frame}
%===========================================================================================================
\begin{frame}[noframenumbering]{Expected upper limit with $(189+18)\invfb$ of data}
\includegraphics[width=0.49\linewidth]{figs/phd_thesis/search_for_b2hnn/fit/limit_vs_syst_Bplus2Kplus.pdf}
\includegraphics[width=0.49\linewidth]{figs/phd_thesis/search_for_b2hnn/fit/limit_vs_syst_Bzero2Kshort.pdf}
\includegraphics[width=0.49\linewidth]{figs/phd_thesis/search_for_b2hnn/fit/CLs_expected_Bplus2Kplus.pdf}
\includegraphics[width=0.49\linewidth]{figs/phd_thesis/search_for_b2hnn/fit/CLs_expected_Bzero2Kshort.pdf}
\end{frame}
%===========================================================================================================
\begin{frame}[noframenumbering]{Uncertainty on $\mu$ with $(63+9)\invfb$ of data}
\bi
\item {\small $\mu = 4.2^{+3.4}_{-3.2} = 4.2^{+2.9}_{-2.8}(\mathrm{stat}){}^{+1.8}_{-1.6}(\mathrm{syst})$.}
\itemii Total uncertainty on $\mu$: profile likelihood scan, fitting the model with fixed values of $\mu$ while keeping the other fit parameters free.
\ei
\vspace{0.5cm}
\centering
\includegraphics[width=0.5\linewidth]{figs/mu_profile.pdf}
\end{frame}
%===========================================================================================================
\begin{frame}[noframenumbering]{Upper limit on $\Br(\BKnn)$}
\bi
\item Upper on the branching fraction determined using the \CLs method.
\itemiii Reminder:
\bi
\itemiii {$\Br(\BKpnn)_{\mathrm{SM}}=\mypow{(4.6\pm0.5)}{-6}$. \hfill \href{https://doi.org/10.1016/j.ppnp.2016.10.001}{\color{blue!40!gray} \tiny [{Prog.~Part.~Nucl.~Phys.~\textbf{92}, 50 (2017)]}}}
\itemiii {$\Br(\BKzznn)_{\mathrm{SM}}=\mypow{(4.3\pm0.5)}{-6}$.}
\ei
\ei
\vspace{0.5cm}

\adjustbox{max height=0.4\textheight,max width=1.0\textwidth}{
\begin{tabular}{llllllll}
Experiment & Year & L [\invfb] & Method & Mode & $\varepsilon_{\mathrm{sig}}$ [\%] & Limit at 90\% CL & Ref.\\
\hline
Babar & 2013 & 429 & HAD & \Kp & 0.04 & $<3.7\times 10^{-5}$ & \href{https://doi.org/10.1103/PhysRevD.87.112005}{\color{blue!40!gray} \tiny Phys.~Rev.~D \textbf{87}, 112005 (2013)} \\
 & & & & \Kz & 0.01 & $<8.1\times 10^{-5}$ & \href{https://doi.org/10.1103/PhysRevD.87.112005}{\color{blue!40!gray} \tiny Phys.~Rev.~D \textbf{87}, 112005 (2013)} \\
Babar & 2013 & 429 & COM & \Kp & - & $<\boldsymbol{1.6\times 10^{-5}}$ & \href{https://doi.org/10.1103/PhysRevD.87.112005}{\color{blue!40!gray} \tiny Phys.~Rev.~D \textbf{87}, 112005 (2013)} \\
 & & & & \Kz & - & $<4.9\times 10^{-5}$ & \href{https://doi.org/10.1103/PhysRevD.87.112005}{\color{blue!40!gray} \tiny Phys.~Rev.~D \textbf{87}, 112005 (2013)} \\
Belle & 2013 & 711 & HAD & \Kp & 0.06 & $<5.5\times 10^{-5}$ & \href{10.1103/PhysRevD.87.111103}{\color{blue!40!gray} \tiny Phys.~Rev.~D \textbf{87}, 111103 (2013)} \\
 & & & & \Kz & 0.004 & $<19\times 10^{-5}$ & \href{10.1103/PhysRevD.87.111103}{\color{blue!40!gray} \tiny Phys.~Rev.~D \textbf{87}, 111103 (2013)} \\
Belle & 2017 & 711 & SL & \Kp & 0.2 & $<1.9\times 10^{-5}$ & \href{https://doi.org/10.1103/PhysRevD.96.091101}{\color{blue!40!gray} \tiny Phys.~Rev.~D \textbf{96}, 091101 (2017)} \\
 & & & & \Kz & 0.05 & $<\boldsymbol{2.6\times 10^{-5}}$ & \href{https://doi.org/10.1103/PhysRevD.96.091101}{\color{blue!40!gray} \tiny Phys.~Rev.~D \textbf{96}, 091101 (2017)} \\
{\color{red} Belle II} & {\color{red} 2021} & {\color{red} 63} & {\color{red} INC} & {\color{red} \Kp} & {\color{red} 4.0} & {\color{red} $<4.1\times 10^{-5}$} & \href{https://doi.org/10.1103/PhysRevLett.127.181802}{\color{red} \tiny Phys.~Rev.~Lett.~\textbf{127}, 181802 (2021)} \\ \\
{\color{gray} Belle II} & {\color{gray} 2022} & {\color{gray} 189} & {\color{gray} INC} & {\color{gray} \Kp} & {\color{gray} 4.0} & {\color{gray} $<1.0\times 10^{-5}$} & {\color{gray} \small (expected)} \\
 & & & & {\color{gray} \Kz} & {\color{gray} 4.0} & {\color{gray} $<3.6\times 10^{-5}$} & {\color{gray} \small (expected)} \\
\hline

\end{tabular}
}

\end{frame}
%===========================================================================================================
\begin{frame}[noframenumbering]{$p_T$ \textit{vs} \qrec}
\centering
\includegraphics[width=0.49\linewidth]{figs/phd_thesis/search_for_b2hnn/first_iteration/Bplus2Kplus_Bsig_H_reconstructed_q2_vs_B_sig_K_pt.pdf}
\end{frame}
%===========================================================================================================
\begin{frame}[noframenumbering]{Post-fit normalisation parameters}
\includegraphics[width=0.49\linewidth]{figs/phd_thesis/search_for_b2hnn/first_iteration/bg_shift.pdf}
\includegraphics[width=0.49\linewidth]{figs/phd_thesis/search_for_b2hnn/first_iteration/corr_data_fit.pdf}
\end{frame}
%===========================================================================================================
\begin{frame}[noframenumbering]{\Bp background in the signal search region}
\centering
\adjustbox{max height=0.4\textheight,max width=1.0\textwidth}{
\begin{tabular}{lrlr}
                   $\tilde{\varepsilon}_{\mathrm{sig}}(\BKpnn)>0.92$ &  Fraction [$\%$] &             $\tilde{\varepsilon}_{\mathrm{sig}}(\BKpnn)>0.98$ &  Fraction [$\%$] \\
\midrule
                    $B^{\pm} \rightarrow D^{0} \mu^{\pm} \nu_{\mu} $ &             13.4 &              $B^{\pm} \rightarrow D^{0} \mu^{\pm} \nu_{\mu} $ &             13.0 \\
          $B^{\pm} \rightarrow D^{*}(2007)^{0} \mu^{\pm} \nu_{\mu} $ &             10.0 &                  $B^{\pm} \rightarrow D^{0} e^{\pm} \nu_{e} $ &              8.2 \\
                        $B^{\pm} \rightarrow D^{0} e^{\pm} \nu_{e} $ &             10.0 &                    $B^{\pm} \rightarrow K^{\pm} K^{0} K^{0} $ &              6.7 \\
              $B^{\pm} \rightarrow D^{*}(2007)^{0} e^{\pm} \nu_{e} $ &              7.2 &    $B^{\pm} \rightarrow D^{*}(2007)^{0} \mu^{\pm} \nu_{\mu} $ &              6.3 \\
                 $B^{\pm} \rightarrow D^{0} e^{\pm} \nu_{e} \gamma $ &              4.6 &                          $B^{\pm} \rightarrow D^{0} K^{\pm} $ &              4.4 \\
                          $B^{\pm} \rightarrow K^{\pm} K^{0} K^{0} $ &              3.6 &           $B^{\pm} \rightarrow D^{0} e^{\pm} \nu_{e} \gamma $ &              4.1 \\
       $B^{\pm} \rightarrow D^{*}(2007)^{0} e^{\pm} \nu_{e} \gamma $ &              3.4 &        $B^{\pm} \rightarrow D^{*}(2007)^{0} e^{\pm} \nu_{e} $ &              3.9 \\
                                $B^{\pm} \rightarrow D^{0} K^{\pm} $ &              2.9 &                $B^{\pm} \rightarrow D^{*}(2007)^{0} K^{\pm} $ &              3.7 \\
                      $B^{\pm} \rightarrow D^{*}(2007)^{0} K^{\pm} $ &              2.6 & $B^{\pm} \rightarrow D^{*}(2007)^{0} e^{\pm} \nu_{e} \gamma $ &              3.6 \\
                         $B^{\pm} \rightarrow \eta_{c}(1S) K^{\pm} $ &              2.4 &                   $B^{\pm} \rightarrow \eta_{c}(1S) K^{\pm} $ &              3.3 \\
                          $B^{\pm} \rightarrow D^{0} K^{\pm} K^{0} $ &              1.7 &                        $B^{\pm} \rightarrow D^{0} \pi^{\pm} $ &              3.2 \\
                              $B^{\pm} \rightarrow D^{0} \pi^{\pm} $ &              1.6 &                  $B^{\pm} \rightarrow \tau^{\pm} \nu_{\tau} $ &              2.0 \\
                        $B^{\pm} \rightarrow \rho(770)^{\pm} D^{0} $ &              1.5 &                          $B^{\pm} \rightarrow K^{0} K^{\pm} $ &              1.7 \\
                  $B^{\pm} \rightarrow D^{0} \tau^{\pm} \nu_{\tau} $ &              1.5 &                  $B^{\pm} \rightarrow \rho(770)^{\pm} D^{0} $ &              1.6 \\
                           $B^{\pm} \rightarrow J/\psi(1S) K^{\pm} $ &              1.4 &                            $B^{\pm} \rightarrow n n K^{\pm} $ &              1.4 \\
%              $B^{\pm} \rightarrow D^{0} \mu^{\pm} \nu_{\mu} \gamma $ &              1.3 &              $B^{\pm} \rightarrow D^{*}(2007)^{0} \pi^{\pm} $ &              1.3 \\
%                     $B^{\pm} \rightarrow D^{*}(2007)^{0} \pi^{\pm} $ &              1.1 &       $B^{\pm} \rightarrow D^{*}(2007)^{0} K^{*}(892)^{\pm} $ &              1.2 \\
%                 $B^{\pm} \rightarrow D^{*}(2007)^{0} K^{\pm} K^{0} $ &              1.1 &       $B^{\pm} \rightarrow D^{0} \mu^{\pm} \nu_{\mu} \gamma $ &              1.2 \\
%         $B^{\pm} \rightarrow D^{*}(2007)^{0} \tau^{\pm} \nu_{\tau} $ &              1.1 &                     $B^{\pm} \rightarrow J/\psi(1S) K^{\pm} $ &              1.2 \\
%           $B^{\pm} \rightarrow D^{0} e^{\pm} \nu_{e} \gamma \gamma $ &              1.1 &           $B^{\pm} \rightarrow f_{2}^{\prime}(1525) K^{\pm} $ &              1.1 \\
%    $B^{\pm} \rightarrow D^{*}(2007)^{0} \mu^{\pm} \nu_{\mu} \gamma $ &              1.1 &                 $B^{\pm} \rightarrow K^{*}(892)^{\pm} D^{0} $ &              1.1 \\
%                             $B^{\pm} \rightarrow D^{0} D_{s}^{\pm} $ &              1.0 &                     $B^{\pm} \rightarrow \phi(1020) K^{\pm} $ &              1.1 \\
%                        $B^{\pm} \rightarrow K^{*}(892)^{\pm} D^{0} $ &              0.9 &    $B^{\pm} \rightarrow D^{0} e^{\pm} \nu_{e} \gamma \gamma $ &              1.0 \\
% $B^{\pm} \rightarrow D^{*}(2007)^{0} e^{\pm} \nu_{e} \gamma \gamma $ &              0.8 &                    $B^{\pm} \rightarrow D^{0} K^{\pm} K^{0} $ &              1.0 \\
%              $B^{\pm} \rightarrow D^{*}(2007)^{0} K^{*}(892)^{\pm} $ &              0.7 &                        $B^{\pm} \rightarrow K^{0} \pi^{\pm} $ &              0.8 \\
%                    $B^{\pm} \rightarrow D_{s}^{*+} D^{*}(2007)^{0} $ &              0.7 &  $B^{\pm} \rightarrow f_{2}^{\prime}(1525) K^{*}(892)^{\pm} $ &              0.8 \\
%               $B^{\pm} \rightarrow D^{*}(2007)^{0} \rho(770)^{\pm} $ &              0.6 &          $B^{\pm} \rightarrow D^{*}(2007)^{0} K^{\pm} K^{0} $ &              0.8 \\
%                                   $B^{\pm} \rightarrow n n K^{\pm} $ &              0.6 &             $B^{\pm} \rightarrow \eta^{\prime}(958) K^{\pm} $ &              0.6 \\
%                         $B^{\pm} \rightarrow \tau^{\pm} \nu_{\tau} $ &              0.5 &                      $B^{\pm} \rightarrow D^{0} D_{s}^{\pm} $ &              0.6 \\
%                   $B^{\pm} \rightarrow D^{*}(2007)^{0} D_{s}^{\pm} $ &              0.5 &                     $B^{\pm} \rightarrow f_{0}(980) K^{\pm} $ &              0.6 \\
\bottomrule

\end{tabular}
}
\end{frame}
%===========================================================================================================
